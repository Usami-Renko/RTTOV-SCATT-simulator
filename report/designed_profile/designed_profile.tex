\documentclass[a4paper]{report}
\usepackage{amsmath}
\usepackage{amssymb} 
\usepackage{graphicx}
\usepackage{subfigure}
\usepackage{geometry}
\usepackage{caption}
\usepackage{floatrow}
\geometry{left=2.0cm, right=2.0cm, top=2.5cm, bottom=3.5cm}
\pagestyle{plain}
\begin{document}

\newcommand{\ud}{\mathrm{d}}
\newcommand{\oops}[1]{\textbf{#1}}

\title{Report of Designed Profile Test}
\author{Hejun Xie\\
Atmospheric Science Department of Zhejiang University}
\date{September 29, 2019}
\maketitle

\newpage
\chapter{Description of RTM Algorithm}

\section{First Principal}
The scalar radiative transfer integro-differential equation is conventionaly
written as below
\footnote{It should be noticed that here we denote the earth surface as $z = 0$, 
and the top of atmosphere (TOA) as $z = z^{*}$}
:  
\begin{eqnarray} \label{eq:firstprincipal}
-\mu \frac{\ud L(z; \mu, \phi)}{k \ud z} & = & 
L(z; \mu, \phi) - J(z; \mu, \phi) \qquad \textrm{(upward)} \nonumber\\
\mu \frac{\ud L(z; -\mu, \phi)}{k \ud z} & = & 
L(z; -\mu, \phi) - J(z; -\mu, \phi) \qquad \textrm{(downward)}
\end{eqnarray}

\oops{where}:
\begin{equation} \label{eq:sourceterm1}
    J(z; \mu, \phi) = 
    \frac{\omega_{0}}{4\pi}\int_{0}^{2\pi}\!\!\!\int_{-1}^{1}
    P(\mu, \phi, \mu', \phi')L(z;\mu',\phi')\ud\mu'\ud\phi'
    + (1-\omega_{0})B(T(z))
\end{equation}
is called source term

\section{Simplification for microwave ratiation tranfer}
\subsection{Angular independency of radiance}
Since the source of microwave radiation is far less anisotropic than solar radiation,
the radiance of microwave radiation is usually treated with no angular denpendency:
the phase function $P(\mu,\phi,\mu',\phi')$ is then reduced to $P(\mu,\mu')$
\begin{eqnarray} \label{eq:angularindependency}
    L(z;\mu,\phi) & = & L(z,\mu) \nonumber\\
    J(z;\mu,\phi) & = & J(z,\mu) \nonumber\\
    P(\mu,\phi,\mu',\phi') & = & \frac{1}{2\pi}\int_{0}^{2\pi}P(\mu,\mu')\ud\phi'
\end{eqnarray}
Insert (\ref{eq:angularindependency}) into (\ref{eq:firstprincipal}), we get:

\begin{eqnarray} \label{eq:firstprincipal2}
\mp\mu \frac{\ud L(z, \pm\mu)}{k \ud z} & = & 
L(z, \pm\mu) - J(z, \pm\mu) \nonumber\\
J(z, \pm\mu) & = &
\frac{\omega_{0}}{2}\int_{-1}^{1}
P(\mu, \pm\mu')L(z,\pm\mu')\ud\mu'
+ (1-\omega_{0})B(T(z))
\end{eqnarray}

\subsection{First Order Spherical Harmonic Method (SHM)}
\subsubsection{Equations}
the first order spherical harmonic method applied in RTTOV-SCATT
is derived from two first order approximations:

\begin{eqnarray} \label{eq:SHMapprox}
L(z,\mu) & = & P_{0}(\mu)L_{0}(z) + P_{1}(\mu)L_{1}(z) 
 =  L_{0}(z) + \mu L_{1}(z) \nonumber\\
P(\cos\Theta) & = & P_{0}(\cos\Theta)P_{0} + P_{1}(\cos\Theta)P_{1}
 =  1 + 3g\cos\Theta
\end{eqnarray}

The scattering angle $\Theta$ can be expressed below:
\begin{displaymath}
    \cos\Theta = \mu\mu' + \sqrt{1-\mu^{2}}\sqrt{1-\mu'^{2}}cos(\phi - \phi')
\end{displaymath}

If plugged into (\ref{eq:angularindependency}), we get the phase function expressed
as first order spherical harmonic expantion:
\begin{equation} \label{eq:pmumup}
    P(\mu,\mu') = \frac{1}{2\pi}\int_{0}^{2\pi}
    \big(
        1 + 3g(\mu\mu' + \sqrt{1-\mu^{2}}\sqrt{1-\mu'^{2}}cos(\phi - \phi'))
    \big)
        \ud\phi'
    = 1 + 3g\mu\mu'
\end{equation}

Then we can express the source term by first order spherical harmonic expansion:
\begin{eqnarray}
J(z,\mu) & = & \frac{\omega_{0}}{2}
\int_{-1}^{1}P(\mu, \mu')L(z, \mu')\ud\mu'
+ (1-\omega_{0})B(T(z))  \nonumber\\
{} & = &  \frac{\omega_{0}}{2}
\int_{-1}^{1}(1+3g\mu\mu')(L_{0}(z) + \mu L_{1}(z))\ud\mu'
+ (1-\omega_{0})B(T(z)))  \nonumber\\
{} & = &  \omega_{0}(g \mu L_{1}(z) + L_{0}(z))
+ (1-\omega_{0})B(T(z))
\end{eqnarray}

Now we finally get the RTM equation in first order SHM approximation 
\begin{equation}
    \mp \mu \frac{\ud L_{0}(z)}{\ud z} - \mu^{2} \frac{\ud L_{1}(z)}{\ud z}
    = L_{0}(z) \pm \mu L_{1}(z) 
    - \omega_{0}(\pm g \mu L_{1}(z) + L_{0}(z))
    - (1-\omega_{0})B(T(z))
\end{equation}

As for the first order SHM method, here we use the first two moment equation straightforward 
instead of the recursive relation of legendre function:

\begin{eqnarray}
    \int_{-1}^{1}\mp \mu \frac{\ud L_{0}(z)}{\ud z} - \mu^{2} \frac{\ud L_{1}(z)}{\ud z} \ud \mu
    & = & \int_{-1}^{1}
    \big(
        L_{0}(z) \pm \mu L_{1}(z) 
    - \omega_{0}(\pm g \mu L_{1}(z) + L_{0}(z))
    - (1-\omega_{0})B(T(z)) 
    \big) \ud \mu \nonumber\\
    \int_{-1}^{1}\mp \mu^{2} \frac{\ud L_{0}(z)}{\ud z} - \mu^{3} \frac{\ud L_{1}(z)}{\ud z} \ud \mu
    & = & \int_{-1}^{1}
    \big(
        \mu L_{0}(z) \pm \mu^{2} L_{1}(z) 
    - \omega_{0}(\pm g \mu^{2} L_{1}(z) + \mu L_{0}(z))
    - (1-\omega_{0})B(T(z)) \mu 
    \big) \ud \mu \nonumber
\end{eqnarray}

Regardless of the sign of $\mu$(i.e., upward or downward), 
we do simplification to the above moment equations and always get the following group
of equation:

\begin{eqnarray} \label{eq:ODEs}
    \frac{\ud L_{0}(z)}{\ud z} & = & -k(z)(1-\omega_{0}(z)g(z))L_{1}(z) \nonumber\\
    \frac{\ud L_{1}(z)}{\ud z} & = & -3k(z)(1-\omega_{0}(z)(L_{0}(z) - B(T(z)))
\end{eqnarray}

Eliminate $L_{1}$ we get the two order ODE

\begin{equation} \label{eq:ODE}
    \frac{\ud^{2}L_{0}(z)}{\ud z ^{2}} = \Lambda(z)^{2}(L_{0}(z) - B(T(z)))
\end{equation}

where:

\begin{equation} \label{eq:Lamda}
    \Lambda(z)^{2} = 3k(z)^{2}(1-\omega_{0}(z)g(z))(1-\omega_{0}(z))
\end{equation}

Equations (\ref{eq:ODE}) and (\ref{eq:Lamda}) is a 
two-order inhomogeneous ordinate differential equation 
with inconstant coefficients, to solve the ODE numerically, 
we cut the interval into layers, and take the approximation that:
\begin{enumerate}
\item The bulk scattering properties such as $k(z), \omega_{0}(z) \textrm{ and } g(z)$
is constant in every single layer
\item The planck function $B(T(z))$ is a linear function of height z in each layer:
(i.e., $B(T(z)) = B_{0} + B_{1}z) $\footnote{Here z denote the height in layer rather than height from earth surface}
\end{enumerate}
With the above assumptions, the question is reduced to a group of 
two-order ODEs with constant coefficients, and a group of join conditions
derived at the boundary of layers(see section boundary conditions).
Then the ODE can be solved analytically in each layer:

\begin{equation} \label{eq:analsolul0}
    L_{0}(z) = D_{+}e^{\Lambda z} + D_{-}e^{ - \Lambda z} + B_{0} + B_{1}z
\end{equation}
and
\begin{equation} \label{eq:analsolul1}
    L_{1}(z) = \frac{-1}{k(1-\omega_{0}g)}\frac{\ud L_{0}(z)}{\ud z} 
    = -\frac{3}{2h} \frac{\ud L_{0}(z)}{\ud z}
\end{equation}
where
\begin{equation} \label{eq:h}
    h = 1.5k(1-\omega_{0}g)
\end{equation}

\subsubsection{Boundary Conditions and Join Conditions}
To derive the boundary conditions and join conditions in the boundary of 
thin layers, we take a different perspective from the spherical moment method in SHM above,
but look at the question in a more phisical way: we care more about the flux density of the radation 
at the layer boundaries. They must be continuous at the boundary.


\begin{eqnarray} \label{eq:fluxdensity} 
    F(z)_{\uparrow}\Big|_{z_{bot}}^{i} & = & F(z)_{\uparrow}\Big|_{z_{top}}^{i+1} \qquad (i=1,2,\ldots,n-1) \nonumber\\
    F(z)_{\downarrow}\Big|_{z_{bot}}^{i} & = & F(z)_{\downarrow}\Big|_{z_{top}}^{i+1} \qquad (i=1,2,\ldots,n-1) \nonumber\\
    F(z)_{\downarrow}\Big|_{z_{top}}^{n} & = & \pi B(T_{cosmos}) \nonumber\\
    F(z)_{\uparrow}\Big|_{z_{bot}}^{1} & = & \overline{\epsilon_{sfc}} \pi B(T_{sfc}) 
    + (1-\overline{\epsilon_{sfc}}) F(z)_{\downarrow}\Big|_{z_{bot}}^{1}  
\end{eqnarray} \footnote{Now, say we cut the profile into n layers, and the layer number increase from surface to cosmos}

Where,
\begin{eqnarray} \label{eq:epsilonbar}
    \overline{\epsilon_{sfc}} & = & 2\int_{0}^{1}\epsilon_{sfc}\mu \ud \mu \nonumber\\
    F(z)_{\uparrow}\Big|_{z_{top/bot}}^{i} & = & \int_{0}^{2\pi}\int_{0}^{1}(L_{0} + \mu L_{1})\mu\ud\mu\ud\phi \nonumber\\
    {} & = & 2\pi\int_{0}^{1}(L_{0}+ \mu L{1})\mu = \pi (L_{0} - \frac{\ud L_{0}}{h \ud z})\Big|_{z_{top/bot}}^{i} \nonumber\\
    F(z)_{\downarrow}\Big|_{z_{top/bot}}^{i} & = & \int_{0}^{2\pi}\int_{-1}^{0}(L_{0} + \mu L_{1})\mu\ud\mu\ud\phi \nonumber\\
    {} & = & \pi (L_{0} + \frac{\ud L_{0}}{h \ud z})\Big|_{z_{top/bot}}^{i}
\end{eqnarray}

Hence,
\begin{eqnarray} \label{eq:boundarycondition} 
    (L_{0} - \frac{\ud L_{0}}{h \ud z})\Big|_{z_{bot}}^{i} & = & (L_{0} - \frac{\ud L_{0}}{h \ud z})\Big|_{z_{top}}^{i+1} \qquad (i=1,2,\ldots,n-1) \nonumber\\
    (L_{0} + \frac{\ud L_{0}}{h \ud z})\Big|_{z_{bot}}^{i} & = & (L_{0} + \frac{\ud L_{0}}{h \ud z})\Big|_{z_{top}}^{i+1} \qquad (i=1,2,\ldots,n-1) \nonumber\\
    (L_{0} + \frac{\ud L_{0}}{h \ud z})\Big|_{z_{top}}^{1} & = & B(T_{cosmos}) \nonumber\\
    (L_{0} - \frac{\ud L_{0}}{h \ud z})\Big|_{z_{bot}}^{n} & = & \overline{\epsilon_{sfc}} B(T_{sfc}) 
    + (1-\overline{\epsilon_{sfc}}) (L_{0} + \frac{\ud L_{0}}{h \ud z})\Big|_{z_{bot}}^{n}  
\end{eqnarray}

Boundary conditions numbers in (\ref{eq:boundarycondition}) add up to 2n, 
thus the group of two order ODE with constant coefficients is well-posed.
That is to say, the question fianlly boil down to solving a group of linear algebra equations, 
whose unknowns are $D_{+}^{i}, D_{-}^{i}(i=1, 2, \ldots, n)$.
It is worth noticing that the coefficients matrix of the boundary condition linear algebra equation
is a band matrix with band width = 4, which can be solved efficiently with LAPACK routine \oops{DGBTRS} and \oops{DGBTRF}.

\subsection{Integrate Source Term}

At the moment, say we have determined the Integrate constant $D_{+}^{i}, D_{-}^{i}(i=1, 2, \ldots, n)$.
Of course, the top of atmosphere radiance (TOA) viewed at any zenith angle $\mu$ can be get simply from
$L(z^{*}, \mu) = L_{0}(z^{*}) + \mu L_{1}(z^{*})$.

However, it must sounds not surprising that such a method is far from precise, since SHM is approximation method
especially in calculating light propogation along a specified ray track. Though it may show good performance
in estimate flux density in solving boundary conditions, since it is a spherical moment method.
\footnote{further comparison between the result given by straight forward and 
integrate source term will be presented in the appendix.}

To derive the integrate source method, we have to start from first principal.
Take the downward equation for instance: 

\begin{eqnarray} \label{eq:downwardeq}
    \mu \frac{\ud L(z, -\mu)}{k \ud z} & = & 
    L(z, -\mu) - J(z, -\mu)
\end{eqnarray}

We do variable seperation and 
integrate (\ref{eq:downwardeq}) it from $\Delta z(z_{top})$ to $0 (z_{bot})$ 

\begin{displaymath}
\int_{\Delta z}^{0} \Big(
    e^{-\frac{k}{\mu}} \ud L(z, -\mu) - \frac{k}{\mu}e^{-\frac{k}{\mu}}L(z, -\mu)\ud z
    \Big) = 
\int_{\Delta z}^{0}\ud\Big(
    e^{-\frac{k}{\mu}}L(z, -\mu)
    \Big) = 
-\int_{\Delta z}^{0}
    \frac{k}{\mu}e^{-\frac{k}{\mu}}J(z, -\mu)\ud z
\end{displaymath}

Hence,

\begin{equation} \label{eq:integratesource}
    L(0, -\mu) = e^{-\frac{kk\Delta z}{\mu}}L(\Delta z, -\mu) + \int_{0}^{\Delta z}
    e^{-\frac{k}{\mu}}J(z, -\mu)\frac{k\ud z}{\mu}
\end{equation}

conventionaly, we call $(1 - e^{-\frac{kz}{\mu}})L(\Delta z, -\mu)$ the \oops{extinction loss
term} in that thin layer, and denote the optical depth by the following expression:

\begin{equation} \label{eq:tau}
    \tau = e^{-\frac{k\Delta z}{\mu}}
\end{equation}

And more, we call the second term on the right hand of (\ref{eq:integratesource}) the \oops{source term} in
in that thin layer: 
\begin{equation} \label{eq:jdo}
    J_{downward} = \int_{0}^{\Delta z}
    e^{-\frac{k}{\mu}z}J(z, -\mu)\frac{k\ud z}{\mu}
\end{equation}

Thus,

\begin{equation} \label{eq:simplified}
    L(z_{bot}, \mu) = \tau L(z_{top}, \mu) + J_{downward} 
    = L(z_{top}, \mu) - L_{extloss} + J_{downward}  
\end{equation}

Likewise, we have

\begin{eqnarray} \label{eq:integratesource2}
    L(\Delta z, \mu) & = & e^{-\frac{k\Delta z}{\mu}}L(0, \mu) + \int_{0}^{\Delta z}
    e^{-\frac{k}{\mu}(\Delta z - z)}J(z, \mu)\frac{k\ud z}{\mu} \qquad \textrm{(upward)}  \nonumber\\
    L(0, -\mu) & = & e^{-\frac{k\Delta z}{\mu}}L(\Delta z, -\mu) + \int_{0}^{\Delta z}
    e^{-\frac{k}{\mu}z}J(z, -\mu)\frac{k\ud z}{\mu} \qquad \textrm{(downward)}
\end{eqnarray}

with

\begin{eqnarray} \label{eq:jspecify}
    J_{downward} & = & \int_{0}^{\Delta z}
    e^{-\frac{k}{\mu(\Delta z - z)}}J(z, -\mu)\frac{k\ud z}{\mu} \nonumber \\
    J_{upward} & = & \int_{0}^{\Delta z}
    e^{-\frac{k}{\mu}z}J(z, -\mu)\frac{k\ud z}{\mu}
\end{eqnarray}

We just need to get $J_{downward}$ and $J_{upward}$ by doing the integration,
for instance:

\begin{displaymath} 
    J_{upward} = \int_{0}^{\Delta z}
    e^{-\frac{k}{\mu}(\Delta z - z)}
    \Big(
        (1-\omega_{0})(B_{0}+B_{1}z) + \omega_{0}(L_{0}(z) + g\mu L_{1}(z)
    \Big)
    \frac{k\ud z}{\mu}
\end{displaymath}


All the preparation have been done, now we can write the source integration over
the track of light as:


\begin{eqnarray} \label{eq:sequential}
    L(z_{top}^{1}, -\mu) & = & B(T_{cosmos}) \nonumber\\
    L(z_{top}^{i+1}/z_{bot}^{i}, -\mu) & = & \tau^{i}L(z_{top}^{i}, -\mu) + J_{down}^{i}(-\mu) 
    \qquad (i=1,2, \ldots, n-1) \nonumber\\
    L(z_{sfc}/z_{bot}^{n}, \mu) & = & \overline{\epsilon_{sfc}}B(T_{sfc}) 
    + (1-\overline{\epsilon_{sfc}})L(z_{sfc}/z_{bot}^{n}, -\mu) \nonumber\\
    L(z_{bot}^{i}/z_{top}^{i+1}, \mu) & = & \tau^{i+1}L(z_{bot}^{i+1}, \mu) + J_{up}^{i+1}(\mu) 
    \qquad (i=1,2, \ldots, n-1) \nonumber\\
    L(z^{*}, \mu) & = & L(z_{top}^{1})
\end{eqnarray}

Equivalently, or, we can track the light in two path:

\begin{enumerate} 
    \item path 1: cosmos - surface - TOA
    \begin{eqnarray} \label{eq:path1rad}
        L(z_{top}^{1}, -\mu) & = & B(T_{cosmos}) \nonumber\\
        L(z_{top}^{i+1}/z_{bot}^{i}, -\mu) & = & \tau^{i}L(z_{top}^{i}, -\mu) + J_{down}^{i}(-\mu) 
        \qquad (i=1,2, \ldots, n-1) \nonumber\\
        L_{path1}(z_{sfc}/z_{bot}^{n}, \mu) & = & (1-\overline{\epsilon_{sfc}})L(z_{sfc}/z_{bot}^{n}, -\mu) \nonumber\\
        L_{path1}(z^{*}, \mu) & = & \tau^{*}L(z_{sfc}/z_{bot}^{n}, \mu)
    \end{eqnarray} 
    where,
    \begin{equation} \label{eq:tau*}
        \tau^{*} = \prod_{i=1}^{n}\tau^{i}
    \end{equation}
    \item path 2: surface - TOA
\end{enumerate}
    \begin{eqnarray} \label{eq:path2rad}
        L_{path2}(z_{sfc}/z_{bot}^{n}, \mu) & = & \overline{\epsilon_{sfc}}B(T_{sfc}) \nonumber\\
        L_{path2}(z_{bot}^{i}/z_{top}^{i+1}, \mu) & = & \tau^{i+1}L_{path2}(z_{bot}^{i+1}, \mu) + J_{up}^{i+1}(\mu) 
        \qquad (i=1,2, \ldots, n-1) \nonumber\\
        L_{path2}(z^{*}, \mu) & = & L_{path2}(z_{top}^{1})
    \end{eqnarray}

Such a way speed up the numerical calculation, and make the analysis of light propogation convenient:
\begin{equation} \label{eq:path1 and path2}
    L(z^{*}, \mu) = L_{path1}(z^{*}, \mu) + L_{path2}(z^{*}, \mu)
\end{equation}

\chapter{Experiment On Designed Profile}

\section{Profile Design Philosophy}
It deserves a lot labour to explain on the motivation to do the profile design experiment,
so as the philosophy of our profile design. Modern numerical weather prediction models still show
poor performance in reproducing the cloud system of convective precipitation events. 
Nevertheless, we have to take a first step in hydrometeor estimation for all-sky assimilation.

Our experiment on designed profile is based on the belief that influence of the vertical inhomogeneity 
of ice particle habit on simulated brightness temperature is nonlinear. By nonlinear, we mean that 
the hydrometeor profile (i.e., the vertical structure of the cloud), is coupled with the vertical inhomogeneity 
of ice particle haibit to dominate the radiative transfer process within the cloud, in a nonlinear way.

Naturally, it comes up to our mind we can design a cloud structure also with two degree of freedom (i.e., 
the low habit layer and high habit layer adjusting factor), to multiply with the mixing ratio profile of
ice hydrometeors and control the cloud structure quantitatively.

To be more specific, we set the standard profile as the eyewall profile of typhoon Feiyan 
simulated by GRAPES model (Fig. \ref{fig:stdprofile}) and set the grid of low and high habit layer factors 
as an array logarithmically evenly spaced between $10^{-2}$ and $10^{1}$ with 40 grids.



\begin{figure}[hb] 
\centering
\includegraphics[width=0.9\textwidth]{./pdf/avgprof.pdf} 
\caption{This figure show the average hydrometeor profile of 
tropical cyclone Feiyan's eyewall, simulated by GRAPES RMFS (10km operational run 2018083100z +3h).
By eyewall, we mean all the profiles that is 35 to 45km away from the storm center.
325hPa is the boundary line of ice particle habit in this experiment}
\label{fig:stdprofile}
\end{figure}

In this research, we get equally interested with the simulated final brightness temperature observarable by satellite,
and also the intermediate radiave tranfer variables like upward and downward \oops{radiance}, \oops{extinction loss} and
\oops{source term} in each RTTOV-SCATT layer (\oops{extinction loss} and \oops{source term}) and at its boundaries (\oops{radiance}), 
as mentioned in (\ref{eq:integratesource}).

\begin{figure}[hb] 
\centering
\includegraphics[width=0.9\textwidth]{./pdf/plotBTexample.pdf}
\caption{This figure show the brightness temperature of four vertical inhomogeneity schemes under various cloud vertical 
structure conditions, quantitatively controled by two factors: low habit layer and high habit layer adjusting factor, 
as x and y axis, each with 40 grids. Here we use the channel 36.5GHZ H of instrument MWRI as example, with standard hydrometeor 
profile as (Fig. \ref{fig:stdprofile}). The zenith of instrument viewing angle is fixed at 60 deg.
Standard profile means high layer adjusting factor and low layer adjusting factor both set as 1.
Simulated by RTTOV-SCATT.}
\label{fig:exBT}
\end{figure}

\begin{figure}[hb] 
\centering
\subfigure[Upward radiance, extinction loss and source term]{
\centering
\includegraphics[width=1.1\textwidth]{./pdf/plotdoradexample.pdf}
\label{subfig:1}
}
\subfigure[Downward radiance, extinction loss and source term]{
\centering
\includegraphics[width=1.1\textwidth]{./pdf/plotupradexample.pdf}
\label{subfig:2}
}
\caption{This figure show the intermediate downward\subref{subfig:1} and upward\subref{subfig:2} radiative transfer variables 
exported from RTTOV-SCATT module by modifying its source code. 
The solid and dotted line in the chart represent the upward or downward rediance 
of different vertical inhomogeneity schemes of ice particle shapes, so with the extinction loss (denoted by solid or hollow circles)
and source term (denoted by solid or hatched bars).
We use solid (positive contribution to radiance) or dashed (negative contribution to radiance) black line segement 
to connect the top of source term bar and the extinction circle, representing the net contribution to radiance in that
RTTOV-SCATT layer. 
Please review (\ref{eq:path1rad} and \ref{eq:path2rad}) to get the exact definition of upward and downward radiance. 
Here we use the channel 89.0GHZ H of instrument MWRI as example, with both high and low cloud structure
adjusting factor as 1. Again, the zenith of instrument viewing angle is set at 60 deg. 
Simulated by RTTOV-SCATT.}
\label{fig:exrad}
\end{figure}

When we look into the intermediate radiation varaibles plotted in (Fig. \ref{fig:exrad}), we find out that the 
downward radiance vary violently from one vertical inhomogeneity schemes to another within one ice-phase layers, but convergent to 
the same value after going through the water-phase layers at the bottom of atmosphere, making the $L_{path1}(z^{*}, \mu)$ radiance 
become insensitive to vertical inhomogeneity schemes. In fact, the eyewall profile always has total transmission $\tau^{*} < 10^{-3}$, 
and that stays true among all the channels from atmosphere windows to the absorption peak of vapour or oxygen, except for 
two low frequency channels: 10.65GHZ and 18.7GHZ of MWRI, which is insensitive to ice pariticle shape. 
So the downward radiance can be neglected when analysing the intermediate radiation varaibles that makes contribution 
to the observarable brightness temperature.

On the other hand, The upward radiance show a different pattern. The radiance stays the same when radiation goes through the water-phase
layers, and begin to diverge significantly when entering the ice-phase layers, since both extinction loss and source term have a big
dynamic range for different vertical inhomogeneity schemes. However, extinction loss and source term vanish in the layers above 125hPa, 
as hydrometeors fade away at tropopause. The conclusion stays true except for the channels set at absorption peak of vapour or oxygen.

Thus, for our exploration of nonlinear radiative transfer process with ice-phase layers, we should focus on the layers between
125hPa and 550hPa, and we zoom in the figure (Fig. \ref{fig:exrad}) to that range in the follwing part of this report.

\clearpage

\section{Bulk Scattering Properties}
Before exploring the radiative transfer process in the cloud, we firstly need to get some impression about the bulk scattering properties 
about the ice particle habits we choose. Also the justification we choose thin plate and dendrite as our experimental shapes for vertical inhomogeneity
test will be shown in this section.

\begin{figure}[hbtp] 
\centering
\includegraphics[width=\textwidth]{./pdf/BSP/against_swc_2column.pdf}
\caption{Bulk scattering properties at 50.3GHZ, ambient temperature at 203K (dashed line, left column) and 273K (solid line, right column).
For thin plate and dendrite, we use Liu (2008) SSP (Single Scattering Properties) optical database,
BSP (Bulk Scattering Properties) have been computed with Field et al. (2007) size distribution.
For Mie spheres, here we use Marshall-Palmer / gamma distribution or Field et al. (2007) size distribution.
We also introduce a shape named aggregate-10-plates from optical database of Bi et al. (2016), which updates the temperature dependance 
of ice refractive indices.}
\label{fig:against_swc_2column}
\end{figure}

Fig. \ref{fig:against_swc_2column} shows the bulk scattering properties of different shapes and PSDs (paritical size distribution) as a
function of snow water contents, at 50.3GHZ. The ambient temperature is varing from 203K to 273K. 

\begin{itemize}
    \item \oops{Extinction} 
    
    Plates and columns tend to have an extinction larger than or about the same (when snow water content is above 0.1 $g \cdot m^{-3}$) as mie sphere when ambient temperature is above 253K,
while the extinction is smaller than mie sphere when temperature falls below 253K. That is to say, the extinction of thin plates increase rapidly with respect to ambient temperature.
Bullets rosettes and snowflakes like dendrite tend to show an extinction always far below mie sphere, especially when ambient temperature is low. (i.e., its extinction
increase not as rapid as thin plate when temperature and snow water content increase.)

    \item \oops{Single Scattering Albedo} 
    
    As for SSA, Plates and columns like thin plate show an even more prominent increase with temperature and snow water content. When temperature is low,
thin plates tend to possess an SSA smaller than mie sphere, but a larger SSA when temperature is high. The SSA of aggregate-10-plates 
always stays below mie sphere, while dendrite snowflake falls far below all other shapes. SSA of both aggregate-10-plates and dendrite
increase significantly with temperature.

    \item \oops{Asymmetry} 
    
    For asymmetry, we can see that miesphere with PSD of both Mashall-Palmer and Field et al. 2007 show a far (only for Mashall Palmer PSD) 
stronger forward scattering than any other aspherical shapes with high snow water content and high ambient temperature. 
It is also interesting that the asymmetry of mie sphere vary little with Mashall-Palmer PSD but significantly with Field et al. 2007 PSD aganist the temperature range.
Asymmetry of aggregate-10-plates is larger than any single crystal shape when temperature and snow water content is high.
the forward scattering of plates and columns is larger than bullets rosettes and snowflakes when temperature is low, but the asymmetry
of dendrite snowflake tends to overtake thin plate when snow water content and temperature is high.
\end{itemize}

The denpendency of temperature for all the bulk scattering properties is important when acounting for the radiative transfer process in the
atmosphere and vertical inhomogeneous ice habit schemes, since the temperature decrease rapidly in ice-phase layers with increasing height.

\begin{figure}[hbtp] 
\centering
\includegraphics[width=0.8\textwidth]{./pdf/BSP/against_swc_1column.pdf}
\caption{Bulk scattering properties at 50.3GHZ, ambient temperature at 203K (dashed line, left column) and 273K (solid line, right column).
We plot the two temperature in one column for convenience of comparison. Shapes of miesphere removed.}
\label{fig:against_swc_1column}
\end{figure}

% \clearpage

Now we move on to the relationship between frequency and bulk scattering property. Pay attention to the fact that the function of bulk scattering properties 
of hydrometeors against frequency must be smooth regardless of the water vapour absorption line or oxygen absorption line
since the bulk scattering table only acount for hydrometeor optical properties. The extinction caused by gas absorption will be added to the layer bulk scattering properties
when all the hydrometeors are added up and then put into the delta-eddington approximation solver of RTTOV-SCATT (see Chapter 1).   

\begin{figure}[hbtp] 
\centering
\includegraphics[width=\textwidth]{./pdf/BSP/against_freq_2column.pdf}
\caption{Bulk scattering properties at 0.1 $g \cdot m^{-3}$, ambient temperature at 203K (dashed line, left column) and 273K (solid line, right column).
We choose 10 channels from MWRI(10.65GHZ, 18.7GHZ, 23.8GHZ, 36.5GHZ, 89.0GHZ), MWTSX(50.3GHZ, 57.29GHZ), MWHSX(118.75$\pm$0.08GHZ, 150, 183.31$\pm$1.0GHZ) aboard on FY-3D
to plot the bulk scattering properties against frequency figure.}
\label{fig:against_freq_2column}
\end{figure}

We tend to see almost the same trend of bulk scattering property in the figure of BSP against frequency spectrum as for 50.3GHZ (Fig. \ref{fig:against_swc_2column} and 
Fig. \ref{fig:against_swc_1column}), since it is a middle frequency channel.
However, there are still some points worth a mention. 

\begin{itemize}
    \item \oops{Extinction} 
    
    For all the shapes and PSDs, extinction increase most significantly in the frequency range of 10GHZ-50GHZ by logarithmical scale.
    Plates and columns, including the aggregate shapes like aggregate-10-plates, show a very high extinction once the frequency reaches as high
    as 150GHZ when temperature is high, leading the mie sphere by half an order of magnitude or so. The bullet rosettes and snowflakes Like
    dendrite possess a low extinction shows a relatively small increase with frequency and temperature in bulk extinction compared with other shapes.
    The extinction of thin plate leads dendrite by one order of magnitude or more for all ambient temperature conditions.  

    \item \oops{Single Scattering Albedo}
     
    The SSA of all shapes and PSD shows a rapid increase with frequency within low frequency region when temperature is high, 
    but increases more smoothly and slowly when temperature is low in full frequency range, except for mie sphere with Mashall-Palmer PSD. 
    The aggregate shapes like aggregate-10-shapes have a higher SSA than thin plate when ambient temperature is low, but things are opposite temperature is high.
    It is well worth noticing that the sharp increase of SSA for thin plate in the region of low-frequecy range, which only takes place when ambient temperature is high.
    For aggregate-10-plate and dendrite snowflake, the increase is more gentle and the slope peak appears in middle and middle-high frequecy range, respectively at 273K.  
    
    \item \oops{Asymmetry}
    
    As for asymmetry, obviously ambient temperature is a dominant factor, especially for high-frequency channels. Mie sphere always have the highest forward scattering ratio
    across the frequency spectrum. It also keeps true for mie sphere with Field et al. 2007 PSD, though the difference is not so huge for high-frequecy channels with low ambient
    temperature. For all the aspherical shapes, an almost homogeneous scattering takes place for low ambient temperature across the frequecy spectrum. But the bulk asymmetry
    of plates and columns, including aggregate shapes, can reach 0.3 ~ 0.4 for highest-frequency channel 183.31$\pm$1.0GHZ, while bullet rosettes and snowflakes like dendrite
    gets no more than 0.1.
    
\end{itemize}


\begin{figure}[hbtp] 
\centering
\includegraphics[width=0.8\textwidth]{./pdf/BSP/against_freq_1column.pdf}
\caption{Bulk scattering properties at 0.1 $g \cdot m^{-3}$, ambient temperature at 203K (dashed line, left column) and 273K (solid line, right column).
We choose 10 channels from MWRI, MWTSX, MWHSX aboard on FY-3D to plot the bulk scattering properties against frequency figure.
We plot the two temperature in one column for convenience of comparison. Shapes of miesphere removed.}
\label{fig:against_freq_1column}
\end{figure}

\clearpage

\section{Exploration of Radiative Transfer Process}

In this section, let us make some insight into the radiative transfer process within the ice-phase layers for 
different vertical inhomogeneity schemes of ice particle shapes.

\subsection{MWRI}
For instrument MWRI, all the channels are set at atmospheric windows.

\subsubsection{10.65GHZ}
For the lowest frequency channel of MWRI, Ice-phase layers give low extinction loss to the radiation and thus different vertical inhomogeneity
schemes make little difference to observarable BT(see Fig. \ref{fig:MWRI1BT}). Its interested target is the large rain drops in low latitude layers.

\begin{figure}[hbtp] 
\centering
\includegraphics[width=0.7\textwidth]{./pdf/MWRI/MWRI1BT.pdf}
\caption{10.65GHZ H brightness temperature over high and low adjusting factor between $10^{-2}$ and $10^{1}$ for four vertical 
inhomogeneity schemes of ice particle shapes.}
\label{fig:MWRI1BT}
\end{figure}

\begin{figure}[hbtp] 
\centering
\subfigure[Full range]{
\centering
\includegraphics[width=\textwidth]{./pdf/MwRI/MWRI1full44.pdf}
\label{subfig:MWRI1full44}
}
\subfigure[Zoom in]{
\centering
\includegraphics[width=0.7\textwidth]{./pdf/MwRI/MWRI1zi44.pdf}
\label{subfig:MWRI1zi44}
}
\caption{10.65GHZ H intermediate radiative transfer variables. $Factor_{high} = Factor_{low} = 10^{1}$.}
\label{fig:MWRI1rad}
\end{figure}

It is worth noticing that for the little difference shown in high factor region for ice phase layers, the shape of low layer ice particles
make more difference than high layer shape. It is also illustrated by Fig. \ref{fig:MWRI1rad} \subref{subfig:MWRI1zi44}.
From Fig. \ref{fig:MWRI1rad} \subref{subfig:MWRI1zi44}, we also notice that the low-layer-shape-dominate phenomenon is mainly 
caused by the fact .


\end{document}
