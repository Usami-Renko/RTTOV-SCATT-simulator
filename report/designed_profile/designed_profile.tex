\documentclass[a4paper]{report}
\usepackage{amsmath}
\usepackage{amssymb} 
\usepackage{graphicx}
\usepackage{geometry}
\geometry{left=2.0cm, right=2.0cm, top=2.5cm, bottom=3.5cm}
\pagestyle{plain}
\begin{document}

\newcommand{\ud}{\mathrm{d}}
\newcommand{\oops}[1]{\textbf{#1}}

\title{Report of Designed Profile Test}
\author{Hejun Xie\\
Atmospheric Science Department of Zhejiang University}
\date{September 29, 2019}
\maketitle

\newpage
\chapter{Description of RTM Algorithm}

\section{First Principal}
The scalar radiative transfer integro-differential equation is conventionaly
written as below
\footnote{It should be noticed that here we denote the earth surface as $z = 0$, 
and the top of atmosphere (TOA) as $z = z^{*}$}
:  
\begin{eqnarray} \label{eq:firstprincipal}
-\mu \frac{\ud L(z; \mu, \phi)}{k \ud z} & = & 
L(z; \mu, \phi) - J(z; \mu, \phi) \qquad \textrm{(upward)} \nonumber\\
\mu \frac{\ud L(z; -\mu, \phi)}{k \ud z} & = & 
L(z; -\mu, \phi) - J(z; -\mu, \phi) \qquad \textrm{(downward)}
\end{eqnarray}

\oops{where}:
\begin{equation} \label{eq:sourceterm1}
    J(z; \mu, \phi) = 
    \frac{\omega_{0}}{4\pi}\int_{0}^{2\pi}\!\!\!\int_{-1}^{1}
    P(\mu, \phi, \mu', \phi')L(z;\mu',\phi')\ud\mu'\ud\phi'
    + (1-\omega_{0})B(T(z))
\end{equation}
is called source term

\section{Simplification for microwave ratiation tranfer}
\subsection{Angular independency of radiance}
Since the source of microwave radiation is far less anisotropic than solar radiation,
the radiance of microwave radiation is usually treated with no angular denpendency:
the phase function $P(\mu,\phi,\mu',\phi')$ is then reduced to $P(\mu,\mu')$
\begin{eqnarray} \label{eq:angularindependency}
    L(z;\mu,\phi) & = & L(z,\mu) \nonumber\\
    J(z;\mu,\phi) & = & J(z,\mu) \nonumber\\
    P(\mu,\phi,\mu',\phi') & = & \frac{1}{2\pi}\int_{0}^{2\pi}P(\mu,\mu')\ud\phi'
\end{eqnarray}
Insert (\ref{eq:angularindependency}) into (\ref{eq:firstprincipal}), we get:

\begin{eqnarray} \label{eq:firstprincipal2}
\mp\mu \frac{\ud L(z, \pm\mu)}{k \ud z} & = & 
L(z, \pm\mu) - J(z, \pm\mu) \nonumber\\
J(z, \pm\mu) & = &
\frac{\omega_{0}}{2}\int_{-1}^{1}
P(\mu, \pm\mu')L(z,\pm\mu')\ud\mu'
+ (1-\omega_{0})B(T(z))
\end{eqnarray}

\subsection{First Order Spherical Harmonic Method (SHM)}
\subsubsection{Equations}
the first order spherical harmonic method applied in RTTOV-SCATT
is derived from two first order approximations:

\begin{eqnarray} \label{eq:SHMapprox}
L(z,\mu) & = & P_{0}(\mu)L_{0}(z) + P_{1}(\mu)L_{1}(z) 
 =  L_{0}(z) + \mu L_{1}(z) \nonumber\\
P(\cos\Theta) & = & P_{0}(\cos\Theta)P_{0} + P_{1}(\cos\Theta)P_{1}
 =  1 + 3g\cos\Theta
\end{eqnarray}

The scattering angle $\Theta$ can be expressed below:
\begin{displaymath}
    \cos\Theta = \mu\mu' + \sqrt{1-\mu^{2}}\sqrt{1-\mu'^{2}}cos(\phi - \phi')
\end{displaymath}

If plugged into (\ref{eq:angularindependency}), we get the phase function expressed
as first order spherical harmonic expantion:
\begin{equation} \label{eq:pmumup}
    P(\mu,\mu') = \frac{1}{2\pi}\int_{0}^{2\pi}
    \big(
        1 + 3g(\mu\mu' + \sqrt{1-\mu^{2}}\sqrt{1-\mu'^{2}}cos(\phi - \phi'))
    \big)
        \ud\phi'
    = 1 + 3g\mu\mu'
\end{equation}

Then we can express the source term by first order spherical harmonic expansion:
\begin{eqnarray}
J(z,\mu) & = & \frac{\omega_{0}}{2}
\int_{-1}^{1}P(\mu, \mu')L(z, \mu')\ud\mu'
+ (1-\omega_{0})B(T(z))  \nonumber\\
{} & = &  \frac{\omega_{0}}{2}
\int_{-1}^{1}(1+3g\mu\mu')(L_{0}(z) + \mu L_{1}(z))\ud\mu'
+ (1-\omega_{0})B(T(z)))  \nonumber\\
{} & = &  \omega_{0}(g \mu L_{1}(z) + L_{0}(z))
+ (1-\omega_{0})B(T(z))
\end{eqnarray}

Now we finally get the RTM equation in first order SHM approximation 
\begin{equation}
    \mp \mu \frac{\ud L_{0}(z)}{\ud z} - \mu^{2} \frac{\ud L_{1}(z)}{\ud z}
    = L_{0}(z) \pm \mu L_{1}(z) 
    - \omega_{0}(\pm g \mu L_{1}(z) + L_{0}(z))
    - (1-\omega_{0})B(T(z))
\end{equation}

As for the first order SHM method, here we use the first two moment equation straightforward 
instead of the recursive relation of legendre function:

\begin{eqnarray}
    \int_{-1}^{1}\mp \mu \frac{\ud L_{0}(z)}{\ud z} - \mu^{2} \frac{\ud L_{1}(z)}{\ud z} \ud \mu
    & = & \int_{-1}^{1}
    \big(
        L_{0}(z) \pm \mu L_{1}(z) 
    - \omega_{0}(\pm g \mu L_{1}(z) + L_{0}(z))
    - (1-\omega_{0})B(T(z)) 
    \big) \ud \mu \nonumber\\
    \int_{-1}^{1}\mp \mu^{2} \frac{\ud L_{0}(z)}{\ud z} - \mu^{3} \frac{\ud L_{1}(z)}{\ud z} \ud \mu
    & = & \int_{-1}^{1}
    \big(
        \mu L_{0}(z) \pm \mu^{2} L_{1}(z) 
    - \omega_{0}(\pm g \mu^{2} L_{1}(z) + \mu L_{0}(z))
    - (1-\omega_{0})B(T(z)) \mu 
    \big) \ud \mu \nonumber
\end{eqnarray}

Regardless of the sign of $\mu$(i.e., upward or downward), 
we do simplification to the above moment equations and always get the following group
of equation:

\begin{eqnarray} \label{eq:ODEs}
    \frac{\ud L_{0}(z)}{\ud z} & = & -k(z)(1-\omega_{0}(z)g(z))L_{1}(z) \nonumber\\
    \frac{\ud L_{1}(z)}{\ud z} & = & -3k(z)(1-\omega_{0}(z)(L_{0}(z) - B(T(z)))
\end{eqnarray}

Eliminate $L_{1}$ we get the two order ODE

\begin{equation} \label{eq:ODE}
    \frac{\ud^{2}L_{0}(z)}{\ud z ^{2}} = \Lambda(z)^{2}(L_{0}(z) - B(T(z)))
\end{equation}

where:

\begin{equation} \label{eq:Lamda}
    \Lambda(z)^{2} = 3k(z)^{2}(1-\omega_{0}(z)g(z))(1-\omega_{0}(z))
\end{equation}

Equations (\ref{eq:ODE}) and (\ref{eq:Lamda}) is a 
two-order inhomogeneous ordinate differential equation 
with inconstant coefficients, to solve the ODE numerically, 
we cut the interval into layers, and take the approximation that:
\begin{enumerate}
\item The bulk scattering properties such as $k(z), \omega_{0}(z) \textrm{ and } g(z)$
is constant in every single layer
\item The planck function $B(T(z))$ is a linear function of height z in each layer:
(i.e., $B(T(z)) = B_{0} + B_{1}z) $\footnote{Here z denote the height in layer rather than height from earth surface}
\end{enumerate}
With the above assumptions, the question is reduced to a group of 
two-order ODEs with constant coefficients, and a group of join conditions
derived at the boundary of layers(see section boundary conditions).
Then the ODE can be solved analytically in each layer:

\begin{equation} \label{eq:analsolul0}
    L_{0}(z) = D_{+}e^{\Lambda z} + D_{-}e^{ - \Lambda z} + B_{0} + B_{1}z
\end{equation}
and
\begin{equation} \label{eq:analsolul1}
    L_{1}(z) = \frac{-1}{k(1-\omega_{0}g)}\frac{\ud L_{0}(z)}{\ud z} 
    = -\frac{3}{2h} \frac{\ud L_{0}(z)}{\ud z}
\end{equation}
where
\begin{equation} \label{eq:h}
    h = 1.5k(1-\omega_{0}g)
\end{equation}

\subsubsection{Boundary Conditions and Join Conditions}
To derive the boundary conditions and join conditions in the boundary of 
thin layers, we take a different perspective from the spherical moment method in SHM above,
but look at the question in a more phisical way: we care more about the flux density of the radation 
at the layer boundaries. They must be continuous at the boundary.


\begin{eqnarray} \label{eq:fluxdensity} 
    F(z)_{\uparrow}\Big|_{z_{bot}}^{i} & = & F(z)_{\uparrow}\Big|_{z_{top}}^{i+1} \qquad (i=1,2,\ldots,n-1) \nonumber\\
    F(z)_{\downarrow}\Big|_{z_{bot}}^{i} & = & F(z)_{\downarrow}\Big|_{z_{top}}^{i+1} \qquad (i=1,2,\ldots,n-1) \nonumber\\
    F(z)_{\downarrow}\Big|_{z_{top}}^{n} & = & \pi B(T_{cosmos}) \nonumber\\
    F(z)_{\uparrow}\Big|_{z_{bot}}^{1} & = & \overline{\epsilon_{sfc}} \pi B(T_{sfc}) 
    + (1-\overline{\epsilon_{sfc}}) F(z)_{\downarrow}\Big|_{z_{bot}}^{1}  
\end{eqnarray} \footnote{Now, say we cut the profile into n layers, and the layer number increase from surface to cosmos}

Where,
\begin{eqnarray} \label{eq:epsilonbar}
    \overline{\epsilon_{sfc}} & = & 2\int_{0}^{1}\epsilon_{sfc}\mu \ud \mu \nonumber\\
    F(z)_{\uparrow}\Big|_{z_{top/bot}}^{i} & = & \int_{0}^{2\pi}\int_{0}^{1}(L_{0} + \mu L_{1})\mu\ud\mu\ud\phi \nonumber\\
    {} & = & 2\pi\int_{0}^{1}(L_{0}+ \mu L{1})\mu = \pi (L_{0} - \frac{\ud L_{0}}{h \ud z})\Big|_{z_{top/bot}}^{i} \nonumber\\
    F(z)_{\downarrow}\Big|_{z_{top/bot}}^{i} & = & \int_{0}^{2\pi}\int_{-1}^{0}(L_{0} + \mu L_{1})\mu\ud\mu\ud\phi \nonumber\\
    {} & = & \pi (L_{0} + \frac{\ud L_{0}}{h \ud z})\Big|_{z_{top/bot}}^{i}
\end{eqnarray}

Hence,
\begin{eqnarray} \label{eq:boundarycondition} 
    (L_{0} - \frac{\ud L_{0}}{h \ud z})\Big|_{z_{bot}}^{i} & = & (L_{0} - \frac{\ud L_{0}}{h \ud z})\Big|_{z_{top}}^{i+1} \qquad (i=1,2,\ldots,n-1) \nonumber\\
    (L_{0} + \frac{\ud L_{0}}{h \ud z})\Big|_{z_{bot}}^{i} & = & (L_{0} + \frac{\ud L_{0}}{h \ud z})\Big|_{z_{top}}^{i+1} \qquad (i=1,2,\ldots,n-1) \nonumber\\
    (L_{0} + \frac{\ud L_{0}}{h \ud z})\Big|_{z_{top}}^{1} & = & B(T_{cosmos}) \nonumber\\
    (L_{0} - \frac{\ud L_{0}}{h \ud z})\Big|_{z_{bot}}^{n} & = & \overline{\epsilon_{sfc}} B(T_{sfc}) 
    + (1-\overline{\epsilon_{sfc}}) (L_{0} + \frac{\ud L_{0}}{h \ud z})\Big|_{z_{bot}}^{n}  
\end{eqnarray}

Boundary conditions numbers in (\ref{eq:boundarycondition}) add up to 2n, 
thus the group of two order ODE with constant coefficients is well-posed.
That is to say, the question fianlly boil down to solving a group of linear algebra equations, 
whose unknowns are $D_{+}^{i}, D_{-}^{i}(i=1, 2, \ldots, n)$.
It is worth noticing that the coefficients matrix of the boundary condition linear algebra equation
is a band matrix with band width = 4, which can be solved efficiently with LAPACK routine \oops{DGBTRS} and \oops{DGBTRF}.

\subsection{Integrate Source Term}

At the moment, say we have determined the Integrate constant $D_{+}^{i}, D_{-}^{i}(i=1, 2, \ldots, n)$.
Of course, the top of atmosphere radiance (TOA) viewed at any zenith angle $\mu$ can be get simply from
$L(z^{*}, \mu) = L_{0}(z^{*}) + \mu L_{1}(z^{*})$.

However, it must sounds not surprising that such a method is far from precise, since SHM is approximation method
especially in calculating light propogation along a specified ray track. Though it may show good performance
in estimate flux density in solving boundary conditions, since it is a spherical moment method.
\footnote{further comparison between the result given by straight forward and 
integrate source term will be presented in the appendix.}

To derive the integrate source method, we have to start from first principal.
Take the downward equation for instance: 

\begin{eqnarray} \label{eq:downwardeq}
    \mu \frac{\ud L(z, -\mu)}{k \ud z} & = & 
    L(z, -\mu) - J(z, -\mu)
\end{eqnarray}

We do variable seperation and 
integrate (\ref{eq:downwardeq}) it from $\Delta z(z_{top})$ to $0 (z_{bot})$ 

\begin{displaymath}
\int_{\Delta z}^{0} \Big(
    e^{-\frac{k}{\mu}} \ud L(z, -\mu) - \frac{k}{\mu}e^{-\frac{k}{\mu}}L(z, -\mu)\ud z
    \Big) = 
\int_{\Delta z}^{0}\ud\Big(
    e^{-\frac{k}{\mu}}L(z, -\mu)
    \Big) = 
-\int_{\Delta z}^{0}
    \frac{k}{\mu}e^{-\frac{k}{\mu}}J(z, -\mu)\ud z
\end{displaymath}

Hence,

\begin{equation} \label{eq:integratesource}
    L(0, -\mu) = e^{-\frac{kk\Delta z}{\mu}}L(\Delta z, -\mu) + \int_{0}^{\Delta z}
    e^{-\frac{k}{\mu}}J(z, -\mu)\frac{k\ud z}{\mu}
\end{equation}

conventionaly, we call $(1 - e^{-\frac{kz}{\mu}})L(\Delta z, -\mu)$ the \oops{extinction loss
term} in that thin layer, and denote the optical depth by the following expression:

\begin{equation} \label{eq:tau}
    \tau = e^{-\frac{k\Delta z}{\mu}}
\end{equation}

And more, we call the second term on the right hand of (\ref{eq:integratesource}) the \oops{source term} in
in that thin layer: 
\begin{equation} \label{eq:jdo}
    J_{downward} = \int_{0}^{\Delta z}
    e^{-\frac{k}{\mu}z}J(z, -\mu)\frac{k\ud z}{\mu}
\end{equation}

Thus,

\begin{equation} \label{eq:simplified}
    L(z_{bot}, \mu) = \tau L(z_{top}, \mu) + J_{downward} 
    = L(z_{top}, \mu) - L_{extloss} + J_{downward}  
\end{equation}

Likewise, we have

\begin{eqnarray} \label{eq:integratesource2}
    L(\Delta z, \mu) & = & e^{-\frac{k\Delta z}{\mu}}L(0, \mu) + \int_{0}^{\Delta z}
    e^{-\frac{k}{\mu}(\Delta z - z)}J(z, \mu)\frac{k\ud z}{\mu} \qquad \textrm{(upward)}  \nonumber\\
    L(0, -\mu) & = & e^{-\frac{k\Delta z}{\mu}}L(\Delta z, -\mu) + \int_{0}^{\Delta z}
    e^{-\frac{k}{\mu}z}J(z, -\mu)\frac{k\ud z}{\mu} \qquad \textrm{(downward)}
\end{eqnarray}

with

\begin{eqnarray} \label{eq:jspecify}
    J_{downward} & = & \int_{0}^{\Delta z}
    e^{-\frac{k}{\mu(\Delta z - z)}}J(z, -\mu)\frac{k\ud z}{\mu} \nonumber \\
    J_{upward} & = & \int_{0}^{\Delta z}
    e^{-\frac{k}{\mu}z}J(z, -\mu)\frac{k\ud z}{\mu}
\end{eqnarray}

We just need to get $J_{downward}$ and $J_{upward}$ by doing the integration,
for instance:

\begin{displaymath} 
    J_{upward} = \int_{0}^{\Delta z}
    e^{-\frac{k}{\mu}(\Delta z - z)}
    \Big(
        (1-\omega_{0})(B_{0}+B_{1}z) + \omega_{0}(L_{0}(z) + g\mu L_{1}(z)
    \Big)
    \frac{k\ud z}{\mu}
\end{displaymath}


All the preparation have been done, now we can write the source integration over
the track of light as:


\begin{eqnarray} \label{eq:sequential}
    L(z_{top}^{1}, -\mu) & = & B(T_{cosmos}) \nonumber\\
    L(z_{top}^{i+1}/z_{bot}^{i}, -\mu) & = & \tau^{i}L(z_{top}^{i}, -\mu) + J_{down}^{i}(-\mu) 
    \qquad (i=1,2, \ldots, n-1) \nonumber\\
    L(z_{sfc}/z_{bot}^{n}, \mu) & = & \overline{\epsilon_{sfc}}B(T_{sfc}) 
    + (1-\overline{\epsilon_{sfc}})L(z_{sfc}/z_{bot}^{n}, -\mu) \nonumber\\
    L(z_{bot}^{i}/z_{top}^{i+1}, \mu) & = & \tau^{i+1}L(z_{bot}^{i+1}, \mu) + J_{up}^{i+1}(\mu) 
    \qquad (i=1,2, \ldots, n-1) \nonumber\\
    L(z^{*}, \mu) & = & L(z_{top}^{1})
\end{eqnarray}

Equivalently, or, we can track the light in two path:

\begin{enumerate} \label{eq:path1}
    \item path 1: cosmos - surface - TOA
    \begin{eqnarray}
        L(z_{top}^{1}, -\mu) & = & B(T_{cosmos}) \nonumber\\
        L(z_{top}^{i+1}/z_{bot}^{i}, -\mu) & = & \tau^{i}L(z_{top}^{i}, -\mu) + J_{down}^{i}(-\mu) 
        \qquad (i=1,2, \ldots, n-1) \nonumber\\
        L_{path1}(z_{sfc}/z_{bot}^{n}, \mu) & = & (1-\overline{\epsilon_{sfc}})L(z_{sfc}/z_{bot}^{n}, -\mu) \nonumber\\
        L_{path1}(z^{*}, \mu) & = & \tau^{*}L(z_{sfc}/z_{bot}^{n}, \mu)
    \end{eqnarray}
    where,
    \begin{equation} \label{eq:tau*}
        \tau^{*} = \prod_{i=1}^{n}\tau^{i}
    \end{equation}
    \item path 2: surface - TOA
\end{enumerate}
    \begin{eqnarray} \label{eq:path2}
        L_{path2}(z_{sfc}/z_{bot}^{n}, \mu) & = & \overline{\epsilon_{sfc}}B(T_{sfc}) \nonumber\\
        L_{path2}(z_{bot}^{i}/z_{top}^{i+1}, \mu) & = & \tau^{i+1}L_{path2}(z_{bot}^{i+1}, \mu) + J_{up}^{i+1}(\mu) 
        \qquad (i=1,2, \ldots, n-1) \nonumber\\
        L_{path2}(z^{*}, \mu) & = & L_{path2}(z_{top}^{1})
    \end{eqnarray}

Such a way speed up the numerical calculation, and make the analysis of light propogation convenient:
\begin{equation} \label{eq:path1 and path2}
    L(z^{*}, \mu) = L_{path1}(z^{*}, \mu) + L_{path2}(z^{*}, \mu)
\end{equation}

\chapter{Expeiment On Designed Profile}


\end{document}

