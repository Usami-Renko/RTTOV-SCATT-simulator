\section{Profile Design Philosophy}
It deserves a lot labour to explain on the motivation to do the profile design experiment,
so as the philosophy of our profile design. Modern numerical weather prediction models still show
poor performance in reproducing the cloud system of convective precipitation events. 
Nevertheless, we have to take a first step in hydrometeor estimation for all-sky assimilation.

The field research \citep*{Bailey09} indicates that ice particle habit has a strong correlation with ambient
temperature and ice supersaturation, which varies significantly with altitude. The habit diagram tends to range 
from polycrystalline to single crystals as temperature ranges from low to high. 
This work together with research of the hydrometeor profile in tropical cyclone at its different development stage \citep*{Zhou18}
give a solid backgroud to the experiment on radiative transfer 
simulation with different vertical inhomogeneous scheme of ice paricles.  

Our experiment on designed profile is based on the belief that influence of the vertical inhomogeneity 
of ice particle habit on simulated brightness temperature is nonlinear. By nonlinear, we mean that 
the hydrometeor profile (i.e., the vertical structure of the cloud), is coupled with the vertical inhomogeneity 
of ice particle haibit to dominate the radiative transfer process within the cloud, in a nonlinear way.

Naturally, it comes up to our mind we can design a cloud structure also with two degree of freedom (i.e., 
the low habit layer and high habit layer adjusting factor), to multiply with the mixing ratio profile of
ice hydrometeors and control the cloud structure quantitatively.

To be more specific, we set the standard profile as the eyewall profile of typhoon Feiyan 
simulated by GRAPES model (Fig. \ref{fig:stdprofile}) and set the grid of low and high habit layer factors 
as an array logarithmically evenly spaced between $10^{-2}$ and $10^{1}$ with 40 grids.

There is an important issue worths your attention for hydrometeor profile design.
Considering that RTTOV-SCATT now employs the following simplification of cloud and precipitation overlap scheme\citep*{Geer09}:

\begin{equation} \label{eq:cloudoverlap}
    cfrac = \frac{\sum_{j}\sum_{i}r_{i}^{j} \Delta z^{j} cc^{j}}
    {\sum_{j}r_{i}^{j} \Delta z^{j}}
\end{equation}

where,

\begin{itemize}
    \item $r_{i}^{j}$ denotes mixing ratio of different type of hydrometeors, 
    (i=rain, snow, cloud water, cloud ice water; j=$1, 2, \cdots, n$), the model have n pressure levels.

    \item $cc^{j}$ denotes the prognostic cloud fraction produced by NWP of j pressure level.

    \item $\Delta z^{j}$ denotes the height of each model pressure level.
\end{itemize}

Then the prognostic mixing ratio averaged over the model grid box, is partitioned into the cloud column,
where $r_{i}^{j}\prime$ denotes the cloud column mixing ratio to be put into Delta-eddington solver:

\begin{equation} \label{eq:partition}
    r_{i}^{j}\prime = \frac{r_{i}^{j}}{cfrac} 
\end{equation}

This approximation makes the assumption that all the cloud and precipitation and concentrated in the cloud fraction.
For the assumed clear-sky part, RTTOV-SCATT uses the RTTOV-DIRECT module to calculate the radiance, and then sums up the cloudy
and clear-sky radiance by weight of cfrac.

\begin{equation} \label{eq:sumup}
    B_{total} = B_{cloudy} \cdot cfrac + B_{clear} \cdot (1 - cfrac)  
\end{equation}

Now comes the \oops{paradox}: if we only change the snow and cloud ice mixing ratio in ice phase layers, the total cfrac
also change (see \ref{eq:cloudoverlap}). That means rain and cloud water mixing ratio in the cloud column also change (see eq:partition),
such a practice makes too many variables in out designed profile experiment and thus should be avoid.

As a solution, we set the cfrac manually as the prescribed value computed with the standard profile (Fig. \ref{fig:stdprofile}) for any designed hydrometeor profile, 
and only get the snow and cloud ice mixing ratio $r_{i}^{j}$ change, so $r_{i}^{j}\prime$ change proportionally.

\begin{figure}[hbtp] 
\centering
\includegraphics[width=0.6\textwidth]{./pdf/avgprof.pdf} 
\caption{This figure show the average hydrometeor profile of 
tropical cyclone Feiyan's eyewall, simulated by GRAPES RMFS (10km operational run 2018083100z +3h).
By eyewall, we mean all the profiles that is 35 to 45km away from the storm center.
325hPa is the boundary line of ice particle habit in this experiment}
\label{fig:stdprofile}
\end{figure}

In this research, we get equally interested with the simulated final brightness temperature observarable by satellite,
and also the intermediate radiave tranfer variables like upward and downward \oops{radiance}, \oops{extinction loss} and
\oops{source term} in each RTTOV-SCATT layer (\oops{extinction loss} and \oops{source term}) and at its boundaries (\oops{radiance}), 
as mentioned in (\ref{eq:integratesource}).

When we look into the intermediate radiation varaibles plotted in (Fig. \ref{fig:exrad}), we find out that the 
downward radiance vary violently from one vertical inhomogeneity schemes to another within one ice-phase layers, but convergent to 
the same value after going through the water-phase layers at the bottom of atmosphere, making the $L_{path1}(z^{*}, \mu)$ radiance 
become insensitive to vertical inhomogeneity schemes. In fact, the eyewall profile always has total transmission $\tau^{*} < 10^{-3}$, 
and that stays true among all the channels from atmosphere windows to the absorption peak of vapour or oxygen, except for 
two low frequency channels: 10.65GHZ and 18.7GHZ of MWRI, which is insensitive to ice particle shape. 
So the downward radiance can be neglected when analysing the intermediate radiation varaibles that makes contribution 
to the observarable brightness temperature.

On the other hand, The upward radiance show a different pattern. The radiance stays the same when radiation goes through the water-phase
layers, and begin to diverge significantly when entering the ice-phase layers, since both extinction loss and source term have a big
dynamic range for different vertical inhomogeneity schemes. However, extinction loss and source term vanish in the layers above 125hPa, 
as hydrometeors fade away at tropopause. The conclusion stays true except for the channels set at absorption peak of vapour or oxygen.

Thus, for our exploration of nonlinear radiative transfer process with ice-phase layers, we should focus on the layers between
125hPa and 550hPa, and we zoom in the figure (Fig. \ref{fig:exrad}) to that range in the follwing part of this report.

\begin{figure}[hbtp] 
\centering
\includegraphics[width=0.72\textwidth]{./pdf/plotBTexample.pdf}
\caption{This figure show the brightness temperature of four vertical inhomogeneity schemes under various cloud vertical 
structure conditions, quantitatively controled by two factors: low habit layer and high habit layer adjusting factor, 
as x and y axis, each with 40 grids. Here we use the channel 183.31$\pm$7.0GHZ of instrument MWHSX as example, with standard hydrometeor 
profile as (Fig. \ref{fig:stdprofile}). The zenith of instrument viewing angle is fixed at 40 deg.
Standard profile means high layer adjusting factor and low layer adjusting factor both set as 1.
Simulated by RTTOV-SCATT.}
\label{fig:exBT}
\end{figure}

\begin{figure}[hbtp] 
\centering
\subfigure[Upward radiance, extinction loss and source term]{
\centering
\includegraphics[width=1.1\textwidth]{./pdf/plotdoradexample.pdf}
\label{subfig:1}
}
\subfigure[Downward radiance, extinction loss and source term]{
\centering
\includegraphics[width=1.1\textwidth]{./pdf/plotupradexample.pdf}
\label{subfig:2}
}
\caption{This figure show the intermediate downward\subref{subfig:1} and upward\subref{subfig:2} radiative transfer variables 
exported from RTTOV-SCATT module by modifying its source code. 
The solid and dotted line in the chart represent the upward or downward rediance 
of different vertical inhomogeneity schemes of ice particle shapes, so with the extinction loss (denoted by solid or hollow circles)
and source term (denoted by solid or hatched bars).
We use solid (positive contribution to radiance) or dashed (negative contribution to radiance) black line segement 
to connect the top of source term bar and the extinction circle, representing the net contribution to radiance in that
RTTOV-SCATT layer. 
Please review (\ref{eq:path1rad} and \ref{eq:path2rad}) to get the exact definition of upward and downward radiance. 
Here we use the channel 89.0GHZ H of instrument MWRI as example, with both high and low cloud structure
adjusting factor as 1. Again, the zenith of instrument viewing angle is set at 60 deg. 
Simulated by RTTOV-SCATT.}
\label{fig:exrad}
\end{figure}

\clearpage
