\chapter{Experiment On Designed Profile}

\section{Profile Design Philosophy}
It deserves a lot labour to explain on the motivation to do the profile design experiment,
so as the philosophy of our profile design. Modern numerical weather prediction models still show
poor performance in reproducing the cloud system of convective precipitation events. 
Nevertheless, we have to take a first step in hydrometeor estimation for all-sky assimilation.

Our experiment on designed profile is based on the belief that influence of the vertical inhomogeneity 
of ice particle habit on simulated brightness temperature is nonlinear. By nonlinear, we mean that 
the hydrometeor profile (i.e., the vertical structure of the cloud), is coupled with the vertical inhomogeneity 
of ice particle haibit to dominate the radiative transfer process within the cloud, in a nonlinear way.

Naturally, it comes up to our mind we can design a cloud structure also with two degree of freedom (i.e., 
the low habit layer and high habit layer adjusting factor), to multiply with the mixing ratio profile of
ice hydrometeors and control the cloud structure quantitatively.

To be more specific, we set the standard profile as the eyewall profile of typhoon Feiyan 
simulated by GRAPES model (Fig. \ref{fig:stdprofile}) and set the grid of low and high habit layer factors 
as an array logarithmically evenly spaced between $10^{-2}$ and $10^{1}$ with 40 grids.

There is an important issue worths your attention for hydrometeor profile design.
Considering that RTTOV-SCATT now employs the following simplification of cloud and precipitation overlap scheme:

\begin{equation} \label{eq:cloudoverlap}
    cfrac = \frac{\sum_{j}\sum_{i}r_{i}^{j} \Delta z^{j} cc^{j}}
    {\sum_{j}r_{i}^{j} \Delta z^{j}}
\end{equation}

where,

\begin{itemize}
    \item $r_{i}^{j}$ denotes mixing ratio of different type of hydrometeors, 
    (i=rain, snow, cloud water, cloud ice water; j=$1, 2, \cdots, n$), the model have n pressure levels.

    \item $cc^{j}$ denotes the prognostic cloud fraction produced by NWP of j pressure level.

    \item $\Delta z^{j}$ denotes the height of each model pressure level.
\end{itemize}

Then the prognostic mixing ratio averaged over the model grid box, is partitioned into the cloud column,
where $r_{i}^{j}\prime$ denotes the cloud column mixing ratio to be put into Delta-eddington solver:

\begin{equation} \label{eq:partition}
    r_{i}^{j}\prime = \frac{r_{i}^{j}}{cfrac} 
\end{equation}

This approximation makes the assumption that all the cloud and precipitation and concentrated in the cloud fraction.
For the assumed clear-sky part, RTTOV-SCATT uses the RTTOV-DIRECT module to calculate the radiance, and then sums up the cloudy
and clear-sky radiance by weight of cfrac.

\begin{equation} \label{eq:sumup}
    B_{total} = B_{cloudy} \cdot cfrac + B_{clear} \cdot (1 - cfrac)  
\end{equation}

Now comes the \oops{paradox}: if we only change the snow and cloud ice mixing ratio in ice phase layers, the total cfrac
also change (see \ref{eq:cloudoverlap}). That means rain and cloud water mixing ratio in the cloud column also change (see eq:partition),
such a practice makes too many variables in out designed profile experiment and thus should be avoid.

As a solution, we set the cfrac manually as the prescribed value computed with the standard profile (Fig. \ref{fig:stdprofile}) for any designed hydrometeor profile, 
and only get the snow and cloud ice mixing ratio $r_{i}^{j}$ change, so $r_{i}^{j}\prime$ change proportionally.

\begin{figure}[hbtp] 
\centering
\includegraphics[width=0.6\textwidth]{./pdf/avgprof.pdf} 
\caption{This figure show the average hydrometeor profile of 
tropical cyclone Feiyan's eyewall, simulated by GRAPES RMFS (10km operational run 2018083100z +3h).
By eyewall, we mean all the profiles that is 35 to 45km away from the storm center.
325hPa is the boundary line of ice particle habit in this experiment}
\label{fig:stdprofile}
\end{figure}

In this research, we get equally interested with the simulated final brightness temperature observarable by satellite,
and also the intermediate radiave tranfer variables like upward and downward \oops{radiance}, \oops{extinction loss} and
\oops{source term} in each RTTOV-SCATT layer (\oops{extinction loss} and \oops{source term}) and at its boundaries (\oops{radiance}), 
as mentioned in (\ref{eq:integratesource}).

When we look into the intermediate radiation varaibles plotted in (Fig. \ref{fig:exrad}), we find out that the 
downward radiance vary violently from one vertical inhomogeneity schemes to another within one ice-phase layers, but convergent to 
the same value after going through the water-phase layers at the bottom of atmosphere, making the $L_{path1}(z^{*}, \mu)$ radiance 
become insensitive to vertical inhomogeneity schemes. In fact, the eyewall profile always has total transmission $\tau^{*} < 10^{-3}$, 
and that stays true among all the channels from atmosphere windows to the absorption peak of vapour or oxygen, except for 
two low frequency channels: 10.65GHZ and 18.7GHZ of MWRI, which is insensitive to ice particle shape. 
So the downward radiance can be neglected when analysing the intermediate radiation varaibles that makes contribution 
to the observarable brightness temperature.

On the other hand, The upward radiance show a different pattern. The radiance stays the same when radiation goes through the water-phase
layers, and begin to diverge significantly when entering the ice-phase layers, since both extinction loss and source term have a big
dynamic range for different vertical inhomogeneity schemes. However, extinction loss and source term vanish in the layers above 125hPa, 
as hydrometeors fade away at tropopause. The conclusion stays true except for the channels set at absorption peak of vapour or oxygen.

Thus, for our exploration of nonlinear radiative transfer process with ice-phase layers, we should focus on the layers between
125hPa and 550hPa, and we zoom in the figure (Fig. \ref{fig:exrad}) to that range in the follwing part of this report.

\begin{figure}[hbtp] 
\centering
\includegraphics[width=0.83\textwidth]{./pdf/plotBTexample.pdf}
\caption{This figure show the brightness temperature of four vertical inhomogeneity schemes under various cloud vertical 
structure conditions, quantitatively controled by two factors: low habit layer and high habit layer adjusting factor, 
as x and y axis, each with 40 grids. Here we use the channel 183.31±7.0GHZ of instrument MWHSX as example, with standard hydrometeor 
profile as (Fig. \ref{fig:stdprofile}). The zenith of instrument viewing angle is fixed at 40 deg.
Standard profile means high layer adjusting factor and low layer adjusting factor both set as 1.
Simulated by RTTOV-SCATT.}
\label{fig:exBT}
\end{figure}

\begin{figure}[hbtp] 
\centering
\subfigure[Upward radiance, extinction loss and source term]{
\centering
\includegraphics[width=1.1\textwidth]{./pdf/plotdoradexample.pdf}
\label{subfig:1}
}
\subfigure[Downward radiance, extinction loss and source term]{
\centering
\includegraphics[width=1.1\textwidth]{./pdf/plotupradexample.pdf}
\label{subfig:2}
}
\caption{This figure show the intermediate downward\subref{subfig:1} and upward\subref{subfig:2} radiative transfer variables 
exported from RTTOV-SCATT module by modifying its source code. 
The solid and dotted line in the chart represent the upward or downward rediance 
of different vertical inhomogeneity schemes of ice particle shapes, so with the extinction loss (denoted by solid or hollow circles)
and source term (denoted by solid or hatched bars).
We use solid (positive contribution to radiance) or dashed (negative contribution to radiance) black line segement 
to connect the top of source term bar and the extinction circle, representing the net contribution to radiance in that
RTTOV-SCATT layer. 
Please review (\ref{eq:path1rad} and \ref{eq:path2rad}) to get the exact definition of upward and downward radiance. 
Here we use the channel 89.0GHZ H of instrument MWRI as example, with both high and low cloud structure
adjusting factor as 1. Again, the zenith of instrument viewing angle is set at 60 deg. 
Simulated by RTTOV-SCATT.}
\label{fig:exrad}
\end{figure}

\clearpage

\section{Bulk Scattering Properties}
Before exploring the radiative transfer process in the cloud, we firstly need to get some impression about the bulk scattering properties 
about the ice particle habits we choose. Also the justification we choose thin plate and dendrite as our experimental shapes for vertical inhomogeneity
test will be shown in this section.

\begin{figure}[hbtp] 
\centering
\includegraphics[width=\textwidth]{./pdf/BSP/against_swc_2column.pdf}
\caption{Bulk scattering properties at 50.3GHZ, ambient temperature at 203K (dashed line, left column) and 273K (solid line, right column).
For thin plate and dendrite, we use Liu (2008) SSP (Single Scattering Properties) optical database,
BSP (Bulk Scattering Properties) have been computed with Field et al. (2007) size distribution.
For Mie spheres, here we use Marshall-Palmer / gamma distribution or Field et al. (2007) size distribution.
We also introduce a shape named aggregate-10-plates from optical database of Bi et al. (2016), which updates the temperature dependance 
of ice refractive indices.}
\label{fig:against_swc_2column}
\end{figure}

Fig. \ref{fig:against_swc_2column} shows the bulk scattering properties of different shapes and PSDs (paritical size distribution) as a
function of snow water contents, at 50.3GHZ. The ambient temperature is varing from 203K to 273K. 

\begin{itemize}
    \item \oops{Extinction} 
    
    Plates and columns tend to have an extinction larger than or about the same (when snow water content is above 0.1 $g \cdot m^{-3}$) as mie sphere when ambient temperature is above 253K,
while the extinction is smaller than mie sphere when temperature falls below 253K. That is to say, the extinction of thin plates increase rapidly with respect to ambient temperature.
Bullets rosettes and snowflakes like dendrite tend to show an extinction always far below mie sphere, especially when ambient temperature is low. (i.e., its extinction
increase not as rapid as thin plate when temperature and snow water content increase.)

    \item \oops{Single Scattering Albedo} 
    
    As for SSA, Plates and columns like thin plate show an even more prominent increase with temperature and snow water content. When temperature is low,
thin plates tend to possess an SSA smaller than mie sphere, but a larger SSA when temperature is high. The SSA of aggregate-10-plates 
always stays below mie sphere, while dendrite snowflake falls far below all other shapes. SSA of both aggregate-10-plates and dendrite
increase significantly with temperature.

    \item \oops{Asymmetry} 
    
    For asymmetry, we can see that miesphere with PSD of both Mashall-Palmer and Field et al. 2007 show a far (only for Mashall Palmer PSD) 
stronger forward scattering than any other aspherical shapes with high snow water content and high ambient temperature. 
It is also interesting that the asymmetry of mie sphere vary little with Mashall-Palmer PSD but significantly with Field et al. 2007 PSD aganist the temperature range.
Asymmetry of aggregate-10-plates is larger than any single crystal shape when temperature and snow water content is high.
the forward scattering of plates and columns is larger than bullets rosettes and snowflakes when temperature is low, but the asymmetry
of dendrite snowflake tends to overtake thin plate when snow water content and temperature is high.
\end{itemize}

The denpendency of temperature for all the bulk scattering properties is important when acounting for the radiative transfer process in the
atmosphere and vertical inhomogeneous ice habit schemes, since the temperature decrease rapidly in ice-phase layers with increasing height.

\begin{figure}[hbtp] 
\centering
\includegraphics[width=0.8\textwidth]{./pdf/BSP/against_swc_1column.pdf}
\caption{Bulk scattering properties at 50.3GHZ, ambient temperature at 203K (dashed line, left column) and 273K (solid line, right column).
We plot the two temperature in one column for convenience of comparison. Shapes of miesphere removed.}
\label{fig:against_swc_1column}
\end{figure}

% \clearpage

Now we move on to the relationship between frequency and bulk scattering property. Pay attention to the fact that the function of bulk scattering properties 
of hydrometeors against frequency must be smooth regardless of the water vapour absorption line or oxygen absorption line
since the bulk scattering table only acount for hydrometeor optical properties. The extinction caused by gas absorption will be added to the layer bulk scattering properties
when all the hydrometeors are added up and then put into the delta-eddington approximation solver of RTTOV-SCATT (see Chapter 1).   

\begin{figure}[hbtp] 
\centering
\includegraphics[width=\textwidth]{./pdf/BSP/against_freq_2column.pdf}
\caption{Bulk scattering properties at 0.1 $g \cdot m^{-3}$, ambient temperature at 203K (dashed line, left column) and 273K (solid line, right column).
We choose 10 channels from MWRI(10.65GHZ, 18.7GHZ, 23.8GHZ, 36.5GHZ, 89.0GHZ), MWTSX(50.3GHZ, 57.29GHZ), MWHSX(118.75$\pm$0.08GHZ, 150, 183.31$\pm$1.0GHZ) aboard on FY-3D
to plot the bulk scattering properties against frequency figure.}
\label{fig:against_freq_2column}
\end{figure}

We tend to see almost the same trend of bulk scattering property in the figure of BSP against frequency spectrum as for 50.3GHZ (Fig. \ref{fig:against_swc_2column} and 
Fig. \ref{fig:against_swc_1column}), since it is a middle frequency channel.
However, there are still some points worth a mention. 

\begin{itemize}
    \item \oops{Extinction} 
    
    For all the shapes and PSDs, extinction increase most significantly in the frequency range of 10GHZ-50GHZ by logarithmical scale.
    Plates and columns, including the aggregate shapes like aggregate-10-plates, show a very high extinction once the frequency reaches as high
    as 150GHZ when temperature is high, leading the mie sphere by half an order of magnitude or so. The bullet rosettes and snowflakes Like
    dendrite possess a low extinction shows a relatively small increase with frequency and temperature in bulk extinction compared with other shapes.
    The extinction of thin plate leads dendrite by one order of magnitude or more for all ambient temperature conditions.  

    \item \oops{Single Scattering Albedo}
     
    The SSA of all shapes and PSD shows a rapid increase with frequency within low frequency region when temperature is high, 
    but increases more smoothly and slowly when temperature is low in full frequency range, except for mie sphere with Mashall-Palmer PSD. 
    The aggregate shapes like aggregate-10-shapes have a higher SSA than thin plate when ambient temperature is low, but things are opposite temperature is high.
    It is well worth noticing that the sharp increase of SSA for thin plate in the region of low-frequecy range, which only takes place when ambient temperature is high.
    For aggregate-10-plate and dendrite snowflake, the increase is more gentle and the slope peak appears in middle and middle-high frequecy range, respectively at 273K.  
    
    \item \oops{Asymmetry}
    
    As for asymmetry, obviously ambient temperature is a dominant factor, especially for high-frequency channels. Mie sphere always have the highest forward scattering ratio
    across the frequency spectrum. It also keeps true for mie sphere with Field et al. 2007 PSD, though the difference is not so huge for high-frequecy channels with low ambient
    temperature. For all the aspherical shapes, an almost homogeneous scattering takes place for low ambient temperature across the frequecy spectrum. But the bulk asymmetry
    of plates and columns, including aggregate shapes, can reach $0.3 \sim 0.4$ for highest-frequency channel 183.31$\pm$1.0GHZ, while bullet rosettes and snowflakes like dendrite
    gets no more than 0.1.
    
\end{itemize}


\begin{figure}[hbtp] 
\centering
\includegraphics[width=0.8\textwidth]{./pdf/BSP/against_freq_1column.pdf}
\caption{Bulk scattering properties at 0.1 $g \cdot m^{-3}$, ambient temperature at 203K (dashed line, left column) and 273K (solid line, right column).
We choose 10 channels from MWRI, MWTSX, MWHSX aboard on FY-3D to plot the bulk scattering properties against frequency figure.
We plot the two temperature in one column for convenience of comparison. Shapes of miesphere removed.}
\label{fig:against_freq_1column}
\end{figure}

\clearpage

\section{Exploration of Radiative Transfer Process}

In this section, let us make some insight into the radiative transfer process within the ice-phase layers for 
different vertical inhomogeneity schemes of ice particle shapes.

\subsection{MWRI}
For instrument MWRI, all the channels are set at atmospheric windows.

\subsubsection{10.65GHZ}
For the lowest frequency channel of MWRI, Ice-phase layers give low extinction loss to the radiation and thus different vertical inhomogeneity
schemes make little difference to observarable BT(see Fig. \ref{fig:MWRI1BT}). Its interested target is the large rain drops in low latitude layers.

\begin{figure}[hbtp] 
\centering
\includegraphics[width=0.7\textwidth]{./pdf/MWRI/MWRI1BT.pdf}
\caption{10.65GHZ H brightness temperature aganist high and low adjusting factor between $10^{-2}$ and $10^{1}$ for four vertical 
inhomogeneity schemes of ice particle shapes.}
\label{fig:MWRI1BT}
\end{figure}

Fig. \ref{fig:MWRI1ABSP44} shows the bulk scattering properties in ice-phase layers acounting for all the hydrometeor types and also the
gas absorption. The vertical gradiant of temperature plays an significant role in this figure. For instance, note that the boundary layer is near 
the peak of mixing ratio of snow particles (see Fig. \ref{fig:stdprofile}) and 450hPa and 275hPa layers have almost the same snow mixing ratio 
and cloud ice mixing ratio, both above the freezing level. But bulk extinction of 450hPa lead its counterpart of 275hPa by an order of magnitude or so
for both shapes as a consequense of the temperature denpendency of snow BSP.
The SSA of thin plate is far larger than dendrite as shown in Fig. \ref{fig:against_freq_1column}. Likewise, the SSA of both shapes drops quickly
for layers above 275hPa because of the mixing ratio and ambient temperature. 
Furthermore, the bulk SSA is reduced by gas absorption when extinction of hydrometeor no longer leads the total extinction.
Both shapes have an quasi-homogeneous scattering since $g < 0.05$ stays true for all ice-phase layers.

\begin{figure}[hbtp] 
\centering
\includegraphics[width=0.8\textwidth]{./pdf/MWRI/MWRI1ABSP44.pdf}
\caption{10.65GHZ H bulk scattering properties acounting for both hydrometeors and gas absorption for ice-phase layers.
$Factor_{upper} = Factor_{lower} = 10^{1}$.}
\label{fig:MWRI1ABSP44}
\end{figure}

Eq. \ref{eq:firstprincipal} and \ref{eq:sourceterm1} show that for low-frequency microwave channels, where water-phase layers have high bulk extinction and low SSA,
and ice-phase layers have low bulk extinction and high SSA, the radiative transfer process is dominated by the net contribution (i.e. source term minus extinction loss term) to 
upward radiance of rain drops at pressure level about 850hPa (illustrated by Fig \ref{fig:MWRI1rad} \subref{subfig:MWRI1full44}). 
The scattering, absorption and emission of ice-phase layers make little difference to the final brightness temperature.

It is worth noticing that for the little difference shown in high factor region for ice phase layers, the shape of low layer ice particles
makes more difference than high layer shape. It is also illustrated by Fig. \ref{fig:MWRI1rad} \subref{subfig:MWRI1zi44}.
From Fig. \ref{fig:MWRI1rad} \subref{subfig:MWRI1zi44}, we also notice that the lower-layer-shape-dominate phenomenon is mainly 
caused by the fact that lower-level are warmer and thus have a higher bulk extinction if ice particle habit and mixing ratio is fixed. 
As dendrite has an lower SSA, it gives more emission.
Naturally, The forward scattering source term minus extinction loss term is negative for quasi-homogeneous scattering.
As shown by Fig. \ref{fig:MWRI1BT} and Fig. \ref{fig:MWRI1rad} \subref{subfig:MWRI1zi44}, dendrite causes about the same scattering loss as emmision contribution
to the upward radiance, while thin plate causes more scattering loss.

Thus, ice-phase layers with dendrite produce a slightly higher (about 0.5 $\sim$ 1K for mixing ratio of $6 \times 10^{-3}$ kg/kg) TOA brightness temperature
than thin plate.

\oops{For low-frequency channels like 10.65GHZ whose ice-phase layers have a low bulk extinction, the superposition effects of vertical inhomogeneous ice habit schemes
is linear (i.e., the TOA radiance variance caused by ice-phase layers can be interpreted as the simple add-up of the solo radiative transfer effects of 
the upper-level and lower-level ice cloud composed by different aspherical shapes of ice crystal), 
since the upward radiance is not dramatically changed after going through lower-level ice cloud.
Things are totally different when it comes to high-frequency microwave regime, 
the nonlinear effect caused by vertical inhomogeneity schemes of ice habit will be shown later.}


\begin{figure}[hbtp] 
\centering
\subfigure[Full range]{
\centering
\includegraphics[width=\textwidth]{./pdf/MwRI/MWRI1full44.pdf}
\label{subfig:MWRI1full44}
}
\subfigure[Ice-phase layers]{
\centering
\includegraphics[width=0.7\textwidth]{./pdf/MwRI/MWRI1zi44.pdf}
\label{subfig:MWRI1zi44}
}
\caption{10.65GHZ H intermediate radiative transfer variables. $Factor_{upper} = Factor_{lower} = 10^{1}$.}
\label{fig:MWRI1rad}
\end{figure}

\clearpage

\subsubsection{18.7GHZ}

The bulk extinction of both shapes at 18.7GHZ leads its counterpart of 10.65GHZ by one order of magnitude as shown by 
Fig. \ref{fig:MWRI2ABSP44} and Fig. \ref{fig:against_freq_2column}. It makes the TOA radiance varaince caused by
ice-phase layers increases significantly (i.e., 10K for mixing ratio of $6 \times 10^{-3}$ kg/kg, see Fig. \ref{fig:MWRI2BT}).
The emission of ice particles can no longer compensate the scattering loss for dendrite at 18.7GHZ.
The lower-layer-dominate phenomenon stays true at 18.7GHZ, a hydrometeor profile with dense lower-layer ice cloud but thin upper-layer
ice cloud give more attenuation to the upward radiance than the opposite.


The variance of TOA upward radiance caused by ice-phase layers can be viewed as a simple e-folding function of the ice water path.
If there are two seperate layers with different shapes, just superposing the seperate e-folding factor gives a good approximation.
The above statement is another expression for linearity of vertical inhomogeneous ice habit scheme.

\begin{figure}[hbtp] 
\centering
\includegraphics[width=0.7\textwidth]{./pdf/MWRI/MWRI2BT.pdf}
\caption{18.7GHZ H brightness temperature against high and low adjusting factor between $10^{-2}$ and $10^{1}$ for four vertical 
inhomogeneity schemes of ice particle shapes.}
\label{fig:MWRI2BT}
\end{figure}

\begin{figure}[hbtp] 
\centering
\includegraphics[width=0.8\textwidth]{./pdf/MWRI/MWRI2ABSP44.pdf}
\caption{18.7GHZ H bulk scattering properties acounting for both hydrometeors and gas absorption for ice-phase layers.
$Factor_{upper} = Factor_{lower} = 10^{1}$.}
\label{fig:MWRI2ABSP44}
\end{figure}

\begin{figure}[hbtp] 
\centering
\includegraphics[width=0.7\textwidth]{./pdf/MwRI/MWRI2zi44.pdf}
\label{fig:MWRI2zi44}
\caption{18.7GHZ H intermediate radiative transfer variables for ice-phase layers. $Factor_{upper} = Factor_{lower} = 10^{1}$.}
\end{figure}

\clearpage


\subsubsection{23.8GHZ, 36.5GHZ and 50.3GHZ}

Here we classify the above three atmospheric window channels from MWRI (23.8GHZ, 36.5GHZ) and MWTSX (50.3GHZ) into the same category 
in consideration of the following reasons:

\begin{itemize}
    \item The parttern of brightness temperature against upper-cloud factor and lower-cloud factor is similar (i.e., there is only quantitative change
like the variance of brightness temperature).

    \begin{table}[htbp]
        \caption{The brightness temperature variance for the channels that approximately preserve the
        so-called linearity of vertical inhomogeneous ice particle schemes in low-frequecy regime.}
        \label{tab:brightnessvariance}
        \centering
        \addtolength{\tabcolsep}{-0mm}
        \begin{tabular}{rccc}
            \toprule[0.75pt]	
            Channel Frequency &  Polarization & BT Variance of Thin Plate & BT Variance of Dendrite \\
            \midrule[0.5pt]	    
            10.65GHZ & V & $1.3$K & $\le 0.1$K \\ 
            10.65GHZ & H & $1.2$K & $\sim$ $0.1$K \\
            18.7GHZ & V & $11.0$K & $\sim$ $0.5$K \\ 
            18.7GHZ & H & $10.6$K & $\sim$ $0.5$K \\
            23.8GHZ & V & $20.7$K & $\sim$ $1$K \\ 
            23.8GHZ & H & $20.7$K & $\sim$ $1$K \\
            36.5GHZ & V & $50.7$K & $\sim$ $5$K \\ 
            36.5GHZ & H & $50.7$K & $\sim$ $5$K \\
            50.3GHZ & V & $58.7$K & $\sim$ $10$K \\  
            \bottomrule[0.75pt]
        \end{tabular}
    \end{table}

    \item The magnitude of bulk extinction of the three channels goes up with frequecy from $10^{-2}$ to $10^{-1}$ $[km^{-1}]$ 
for snow water mixing ratio of $6 \times 10^{-3}$ kg/kg. The gap of bulk extinction bwtween the two shapes get wider with increasing frequency,
while that of bulk SSA narrowed, since the SSA of dendrite increases with frequency steadily but that of thin plate has hit the ceiling of 1, 
see Fig. \ref{fig:against_freq_2column}). Just mention that the homogeneous scattering approximation is getting invalid 
with increasing frequency when bulk asymmetry hits the threshold of 0.1. 
Dendrite has a higher bulk asymmetry where the mixing ratio and ambient temperature is high, vice versa
(also illustrated by Fig. \ref{fig:against_freq_2column}).
\end{itemize}

The linearity of vertical inhomogeneous ice habit scheme keeps its validity for this region. Everything mentioned in section 18.7GHZ can be 
simply transplanted to this section to give a analysis to the radiative tranfer process.
Here I also want to give an example to expand on how the linearity begin to crash for high ice cloud factor at channel 50.3GHZ.

Here we just enumerate the phenomena that suggest the linearity begin to crash, the radiative transfer mechanisms contribute
to them will be elaborated on latter in section 89.0GHZ, where the following phenomena get even more prominent.

\begin{itemize}
    \item The upward radiance has nearly reduced by half after going through the lower ice-phase layer (see Fig. \ref{fig:MWTSX1zi44}).
    \item The upward radiance of Thin plate and 'Dendrite / Thin plate' (Dendrite and 'Thin plate / Dendrite', too) 
begins to bifurcate significantly before the boundary layer at 325hPa (see Fig. \ref{fig:MWTSX1zi44}),
even though that pair of profile have the same bulk scattering properties below the shape boundary line.
    \item The Dendrite scheme attenuates the upward radiance above the shape boundary line, but the 'Dendrite / Thin plate'
scheme intensify the upward radiance slightly in the same layer (see Fig. \ref{fig:MWTSX1zi44}).
    \item The brightness temperature contour line slightly goes weirdly upward at the top right corner for the 'Dendrite / Thin plate' scheme
    (see Fig. \ref{fig:MWTSX1BT}). 
\end{itemize}

\begin{figure}[hbtp] 
\centering
\includegraphics[width=0.7\textwidth]{./pdf/MWTSX/MWTSX1BT.pdf}
\caption{50.3GHZ brightness temperature aganist high and low adjusting factor between $10^{-2}$ and $10^{1}$ for four vertical 
inhomogeneity schemes of ice particle shapes.}
\label{fig:MWTSX1BT}
\end{figure}

\begin{figure}[hbtp] 
\centering
\includegraphics[width=0.7\textwidth]{./pdf/MwTSX/MWTSX1zi44.pdf}
\caption{50.3GHZ intermediate radiative transfer variables for ice-phase layers, where $Factor_{upper} = Factor_{lower} = 10^{1}$.}
\label{fig:MWTSX1zi44}
\end{figure}

\clearpage

\subsubsection{89.0GHZ}

Before involving with the complicated non-linear effects of radiative transfer process at 89.0GHZ, let we review 
a few concepts and terms concerning the radiative transfer algorithm described in chapter 1 and introduce something new.

the source term of a layer can be seperated into the sum of \oops{emission source term} and \oops{scattering source term}:

\begin{equation} \label{eq:seperatesourceterm} 
    J_{\uparrow} = \int_{0}^{\Delta z}
    e^{-\frac{k}{\mu}(\Delta z - z)}
    \Big(
        (1-\omega_{0})(B_{0}+B_{1}z) + \omega_{0}(L_{0}(z) + g\mu L_{1}(z)
    \Big)
    \frac{k\ud z}{\mu} 
    = J_{\uparrow ems} + J_{\uparrow sca} 
\end{equation}

where,

\begin{eqnarray} \label{eq:emsandscaexpression}
J_{\uparrow ems} & = & (1-\omega_{0})\int_{0}^{\Delta z}
e^{-\frac{k}{\mu}(\Delta z - z)}
    (B_{0}+B_{1}z)
\frac{k\ud z}{\mu} \nonumber\\
J_{\uparrow sca} & = & \omega_{0}\int_{0}^{\Delta z}
e^{-\frac{k}{\mu}(\Delta z - z)}
(L_{0}(z) + g\mu L_{1}(z))
\frac{k\ud z}{\mu}
\end{eqnarray}

Pay attention to the fact that $J_{\uparrow ems}$ is an \oops{intrinsic} term but $J_{\uparrow sca}$ is \oops{extrinsic}.
By intrinsic, we indicate that this term is irrelavant with the radiative enviroment it is placed in, but only concerns
the material variables in the atmosphere like temperature and mixing ratio of hydrometeor and its shape and so on.
On the contrary, the scattering term is substantially affected by the radiance from neighboring layers.

Apply the above definition to the \oops{extinction loss term}, obviously we find extinction loss term is an extrinsic term but with
an intrinsic factor: $1 - e^{-\frac{k\Delta z}{\mu}}$. That is, the extinction loss term grows proportionally with the upward radiance if
the bulk scattering properties in that layer are fixed.

\begin{equation} \label{eq:extlossterm}
L_{\uparrow extloss} = (1 - e^{-\frac{k\Delta z}{\mu}})L(0, \mu)
\end{equation} 

We call $L_{0}(z)$ the \oops{isotropy factor of radiance}, and $L_{1}(z)$ the \oops{anisotropy factor of radiance}.
For the sake of concise, we can use the following approximation to replace $L_{0}(z)$ and $L_{1}(z)$ the when reading the 
intermediate radiative transfer charts.

\begin{eqnarray} \label{eq:approxforLupLdo}
    L_{\uparrow}(z)   \qquad \textrm{or} \qquad L(z, \mu_{0})   & = & L_{0}(z) + \mu_{0}L_{1}(z) \nonumber\\
    L_{\downarrow}(z) \qquad \textrm{or} \qquad L(z, -\mu_{0})  & = & L_{0}(z) - \mu_{0}L_{1}(z)
\end{eqnarray}

From above we get a rough estimation of  $L_{0}(z)$ and $L_{1}(z)$, where $\mu_{0}$ is the cosine of instrument viewing zenith.

\begin{eqnarray} \label{eq:approxforL0L1}
    L_{0}(z)     & = & \frac{L_{\uparrow}(z) + L_{\downarrow}(z)}{2} \nonumber\\
    L_{1}(z)     & = & \frac{L_{\uparrow}(z) - L_{\downarrow}(z)}{2\mu_{0}}
\end{eqnarray}

The relationship lies between single scattering albedo and $J_{\uparrow ems}$ or $J_{\uparrow sca}$ is easy to derive:
the scattering source term grows and the emmision source term decreases with SSA, both linearly. Yet, the comprehensive
effect of SSA on total source term is hard to evaluate. But there are some useful conclusions easily to access:

\begin{enumerate}
    \item \oops{proposition 1.} If the material variables in a layer is fixed, an intense radiation from neighboring layers (i.e., large $L_{0}(z)$ and $L_{1}(z)$)
increases the extrinsic term significantly, but leave the intrinsic terms unchanged. To put it quantitatively, if the bulk asymmetry is negligible,
than both extinction loss term and scattering source term grows proportionally with the upward radiance (Eq. \ref{eq:extlossterm} and \ref{eq:emsandscaexpression}).
On the other hand, the emmision source term keeps constant. Since extinction loss term is always larger than scattering source term for homogeneous scattering if $\omega_{0} < 1$,
than there is a certain value of upward radiance that equilibrate the emmision, scattering and extinction. The net contribution of that layer to the upward radiance
is negative if the upward radiance is less than that value, vice versa.  
    \item \oops{proposition 2.} If the enviroment of radiance is fixed and anisotropic, say, upward radiance is much stronger than downward radiance,
    which is often the case in ice-phase layers, positive $L_{1}(z)$ is garanteed. Than we find bigger asymmetry will make 
    scattering source term larger as indicated by Eq. \ref{eq:emsandscaexpression}.
    \item \oops{proposition 3.} If upward radiance is fixed and only downward radiance varies, that means if $L_{\downarrow}(z)$ gets bigger, $L_{0}(z)$
    increases but $L_{1}(z)$ decreases. $L_{0}(z) + g\mu L_{1}(z)$ eventually increases as $g < 1$ always establishes, that makes scattering
    source term increases (see Eq. \ref{eq:emsandscaexpression}).  
\end{enumerate}

\begin{figure}[hbtp] 
\centering
\subfigure[Upward radiance, source term and extinction loss term]{
\centering
\includegraphics[width=0.70\textwidth]{./pdf/MWRI/MWRI5zi33up.pdf}
\label{subfig:MWRI5zi33up}
}
\subfigure[Downward radiance, source term and extinction loss term]{
\centering
\includegraphics[width=0.70\textwidth]{./pdf/MWRI/MWRI5zi33do.pdf}
\label{subfig:MWRI5zi33do}
}
\caption{Intermediate radiative transfer variables of standard hydrometeor profile at channel 89.0GHZ H: $factor_{lower} = factor_{lower} = 10^{0}$.}
\label{fig:MWRI5zi33}
\end{figure}

Now we can take a thorough inspection about the the wierd non-linear radiative transfer process at 89.0GHZ.

As shown by Fig. \ref{fig:MWRI5ABSP}, the bulk extinction of layers using thin plates leads its counterpart of dendrite
that both computed from the same mixing ratio of hydrometeors by one order of magnitude. And the gap gets wider if the mixing ratio
reaches 10 times of the standard eyewall profile (Fig. \ref{fig:MWRI5ABSP} \subref{subfig:MWRI5ABSP44}). Also, the bulk SSA of
layers using thin plates also overwhelms dendrite, though the difference is narrowed with extremely high mixing ratio of hydrometeors. 
The bulk asymmetry of both shapes is close at lower layers with high mixing ratio, say about 0.2.

Then we take a look at the intermediate radiative transfer variables shown in Fig. \ref{fig:MWRI5zi33} and \ref{fig:MWRI5zi44}.

\begin{itemize}
    \item \oops{Phenomenon 1.}
    Though the bulk scattering properties of Thin plate and 'Dendrite / Thin plate' is exactly the same for layers below the
boundary line of ice particle habit, the latter has a more sudden drop of upward radiance near the that line, so does for the pair
of 'Thin plate / Dendrite' and Dendrite (Fig. \ref{fig:MWRI5zi33}). The phenomenon gets more prominent for extremely high mixing ratio
of hydrometeors, especially the the upward radiance of thin plate has a lifting in layers below boundary (Fig. \ref{fig:MWRI5zi44}).

    \oops{Mechanism}: As proposed in \oops{proposition 3}, the enhancement of downward radiance tends to increase the scattering source term 
significantly. Take the pair of Thin plate and 'Dendrite / Thin plate' for example, the downward radiance of Thin plate lead
that of 'Dendrite / Thin plate' by one order of magnitude because of the high bulk extinction of Thin plate in the layers above boundary.
The downward radiance at the the bottom of ice particle habit layer of thin plate is comparable ($10^{-2}$ $mW \cdot cm \cdot sr \cdot m^{2}$) 
with the upward radiance at that height, thus should increase anisotropic factor $L_{0}$ significantly (Fig. \ref{fig:MWRI5zi33} \subref{subfig:MWRI5zi33do}).
We refer to the above mechanism as \oops{coupling effect of downward and upward radiance near habit boundary line}.
The variance of downward radiance influences the radiative tranfer process of upward radiance in that coupling way 
rather than affects the TOA radiance by surface reflectance, for high frequencies.
The variance of scattering source term also have a positive correlation with the bulk SSA and extinction of the lower habit layer
(Eq. \ref{eq:emsandscaexpression}). So thin plate benefits from that mechanism more than 'Thin plate / Dendrite' and that explains the lifting
in layers below boundary.

    \item \oops{Phenomenon 2.}
    Though the bulk scattering properties of Thin plate and 'Thin plate / Dendrite' is exactly the same for layers above the
boundary line of ice particle habit, the upward radiance of 'Thin plate / Dendrite' is almost not attenuated by upper habit layer
as that of Dendrite does. 

    \oops{Mechanism}: As proposed in \oops{proposition 1}, if the bulk scattering property at a certain layer is fixed, the intensity
of upward radiance determines the sign of net contribution of that layer to it for quasi-homogeneous scattering. 
As the upward radiance is reduced more than half by lower layer with high bulk extinction 
(Fig. \ref{fig:MWRI5zi33} \subref{subfig:MWRI5zi33up}. The upward radiance must have been reduced to the 'equilibrium threshold' 
for 'Dendrite / Thin plate'. That makes the upward radiance not affected by the upper habit layer.
We refer to the above mechanism as \oops{equilibrium effect above high bulk extinction layer}.
\end{itemize}

Pay attention to the fact that both of the above effect is called \oops{non-linear effect}, for both of them are caused by the prominent
change of upward or downward radiance after going through ice-phase layers. You may draw an analogy to non-linearity arises in  
the vibrance of the string if its amplitude exceeds the threshold of assumption to derive the linear wave equation.

\begin{figure}[hbtp] 
\centering
\subfigure[Upward radiance, source term and extinction loss term]{
\centering
\includegraphics[width=0.70\textwidth]{./pdf/MWRI/MWRI5zi44up.pdf}
\label{subfig:MWRI5zi44up}
}
\subfigure[Downward radiance, source term and extinction loss term]{
\centering
\includegraphics[width=0.70\textwidth]{./pdf/MWRI/MWRI5zi44do.pdf}
\label{subfig:MWRI5zi44do}
}
\caption{Intermediate radiative transfer variables of extremely high mixing ratio hydrometeor profile
at channel 89.0GHZ H: $factor_{lower} = factor_{lower} = 10^{1}$.}
\label{fig:MWRI5zi44}
\end{figure}

\begin{figure}[hbtp] 
\centering
\subfigure[$factor_{lower} = factor_{lower} = 10^{0}$]{
\centering
\includegraphics[width=0.63\textwidth]{./pdf/MWRI/MWRI5ABSP33.pdf}
\label{subfig:MWRI5ABSP33}
}
\subfigure[$factor_{lower} = factor_{lower} = 10^{1}$]{
\centering
\includegraphics[width=0.63\textwidth]{./pdf/MWRI/MWRI5ABSP44.pdf}
\label{subfig:MWRI5ABSP44}
}
\caption{Total bulk scattering properties accounting for hydrometeors and gas at channel 89.0GHZ H.}
\label{fig:MWRI5ABSP}
\end{figure}

\begin{figure}[hbtp] 
\centering
\includegraphics[width=0.68\textwidth]{./pdf/MWRI/MWRI5BT.pdf}
\caption{89.0GHZ H brightness temperature aganist high and low adjusting factor between $10^{-2}$ and $10^{1}$ for four vertical 
inhomogeneity schemes of ice particle shapes.}
\label{fig:MWRI5BT}
\end{figure}

\begin{figure}[hbtp] 
\centering
\includegraphics[width=0.8\textwidth]{./pdf/MWRI/MWRI5OVB.pdf}
\caption{Observed and simulated over-ocean brightness temperatures on 31 August 2018 in the region of Typhoon Feiyan at channel 89.0GHZ H of MWRI.
Simulations are generated by GRAPES RMFS forecast (10km operational run 2018083100z +4h). The satellite FY-3D overpassed the
tropical cyclone at 04:02 UTC}
\label{fig:MWRI5OVB}
\end{figure}

Now let us focus on analysing the wierd pattern shown in Fig \ref{fig:MWRI5OVB} and the shielding effect indicated by it.
\begin{itemize}
    \item \oops{Phenomenon 3.}
    For ice particle habit scheme 'Dendrite / Thin plate', its simulated brightness temperatures show an opposite trend when 
lower-layer mixing ratio is high and low. That is, the brightness temperature decreases with the higher-layer hydrometeor mixing ratio,    
when lower-layer hydrometeor mixing ratio is low, but increases with it when lower-layer hydrometeor mixing ratio is extremely high
(Fig. \ref{fig:MWRI5BT}).

    \oops{Mechanism}:
    The upward radiance is reduced to its half at 500hPa after going through thin plate layer for 'Dendrite / Thin plate'. Remind that
the scattering source term and extinction loss term is both quasi-proportional to the upward radiance, so the variance of scattering source
term and extinction loss term concerning the mixing ratio of upper-layer hydrometeor mixing ratio (i.e., the bulk extinction and SSA of that)
is relatively small and so is the net contribution to upward radiance. Also the order of bulk extinction and SSA of dendrite is low by itself.
The above reasons makes the attenuation effect of upper layer cannot predominate the following secondary factor. 
    The increase of mixing ratio of hydrometeor of upper layer enhances the downward radiance significantly. Also, the lower layer is composed of
thin plate snow crystals, which garantee high bulk extinction and SSA. The increasing downward radiance near the boundary line dramatically boosts the
scattering source term of the layer below it (Eq. \ref{eq:emsandscaexpression}). Eventually, we find the secondary opposite factor predominate
the process when mixing ration of hydrometeor of lower layer is extremely high.   
    The above mechanism acounts for the abnormal pattern at top right corner of the figure.
    
    \item \oops{Phenomenon 4.}
    The gradient of brightness temperature against upper or lower mixing ratio of hydrometeors decreases significantly when mixing ratio is extremely high
for scheme 'Thin plate'(Fig. \ref{fig:MWRI5BT}). 

    \oops{Mechanism}:
    The downward radiance of scheme 'Thin plate' tends to hit a ceiling at a certain optical depth where the downward radiance is
large enough to balance emmision, scattering end extinction, so that the upward radiance keeps constant after going though that layer (Fig. \ref{fig:MWRI5zi44}).
That means the TOA brightness temperature is irrelavant with the optical depth below that level, as is often the case in optical-thick remote sensing.
We will refer to that phenomenon as \oops{sheilding}.

    \item \oops{Phenomenon 5.}
    For ice particle habit scheme 'Thin plate / Dendrite', the upper layer is able to shield any lower layer mixing ratio ($10^{-2} \sim 10^{1}$) again 
when the higher layer hydrometeor factor is slightly larger than $10^{0}$ (Fig. \ref{fig:MWRI5BT}).

    \oops{Mechanism}:
    The attenuation of upward radiance by dendrite layer is not large enough to trigger the non-linear effect.  
When the density of upper thin plate layer is high, it tends to enhance the downward radiance significantly.
Meanwhile, when the lower layer density grows, it again has a direct factor and secondary factor of net contribution to the upward radiance:
    \oops{Direct factor}: The growing mixing ratio of lower layer tends to increase the amplitude of attenuation to the upward radiance since the upward radiance 
is at its peak.
    \oops{Secondary factor}: The growing bulk extinction and SSA of lower layer tends to increase its responding benefit from the intense downward radiance from upper layer.
    Now we see the direct factor tends to dominate the secondary factor at the top right corner of the figure (Fig. \ref{fig:MWRI5BT}). 
And the upper layer is slightlt shielding the lower layer mixing ratio in this region.
\end{itemize}

\oops{Generalization for shielding effect}: From the above analysis, we find that the habit layer with higher extiction and SSA tends shield the mixing ratio of another habit layer, no matter
it is in the upper layer or lower layer. 
    For scheme 'Dendrite / Thin plate', we classify the shield effects into \oops{under-shielding}, \oops{perfect-shielding} and \oops{over-shielding} for different
lower layer mixing ratio factor of $10^{-2} \sim 10^{0}$, $10^{0}$ and $10^{0} \sim 10^{1}$, respectively. We may also refer to the effect in sheme 'Thin plate / Dendrite'
as under-sheilding, of course.

    The shielding effect plays an important role in the forward simulation of real-world cloud and precipitation system (Fig. \ref{fig:MWRI5OVB}).

\begin{itemize}
    \item \oops{The sensitive layer}: 
    The sheilding effect makes the detail or texture displayed in the simulated brightness temperature tends to reveal the mixing ratio of different layers, given that
the real-world profile is concentrated near $factor_{lower} = factor_{lower} = 10^{0}$.
    For 'Thin plate', it tends to display the information in both layers, i.e. the optical depth of ice-phase layer, though a ceilng exsits for shielding effect. 
    For 'Dendrite', it tends to display the optical depth of ice-phase layer, with a small scale, though.
    For 'Thin plate / Denrite', it primarily tends to display the upper layer mixing ratio for the under-shielding effect.
    For 'Denrite / Thin plate', it primarily display the lower layer mixing ratio, with slight opposite adjustment to it by upper layer mixing ratio, for different
lower layer mixing ratio.
    \item \oops{The scale and offset}:
The scale and offset to display the mixing ratio information by brightness temperature is also varying for different schemes. For instance, the scale is larger for 
'Thin plate / Denrite' than 'Denrite / Thin plate' as the bulk extinction is higher in lower layer becasuse of higher ambient temperature.  
\end{itemize}

\clearpage

\subsection{MWHSX}
Instrument MWHSX has two atmospheric window channels, namely 89.0GHZ and 150.0GHZ, and two series of vapour sounding channels
sharing the same central frequency at water vapor absorption lines (i.e., 118.75GHZ and 183.31GHZ)
but with different width of sidebands, on purpose of detecting the water vapour mixing ratio at various altitudes.

We try to look at the quasi-window channels (118.75$\pm$5.0GHZ and 183.31$\pm$7.0GHZ) together with window channnels 
(89.0GHZ and 150.0GHZ) first. We denote the former as 118.75GHZ and 183.31GHZ in the follwing chapter for convenience before further notice.  
At 118.75GHZ, things like shielding effects are quite the same as 89.0GHZ for different vertical schemes of ice habit, only quantitative differences found.
But the lower-layer shielding effect for scheme 'Dendrite / Thin plate' gets invalid when upper layer gets optically thick enough
at channel 183.31GHZ, which will be interpreted in next section. So we choose the two channels: 150.0GHZ and 183.31GHZ
to elaborate on the radiative transfer process within high-frequency microwave regime. 

\subsubsection{150.0GHZ and 183.31GHZ}

\begin{figure}[hbtp] 
\centering
\subfigure[Upward radiance, source term and extinction loss term]{
\centering
\includegraphics[width=0.70\textwidth]{./pdf/MWHSX/MWHSX2zi44up.pdf}
\label{subfig:MWHSX2zi44up}
}
\subfigure[Downward radiance, source term and extinction loss term]{
\centering
\includegraphics[width=0.70\textwidth]{./pdf/MWHSX/MWHSX2zi44do.pdf}
\label{subfig:MWHSX2zi44do}
}
\caption{Intermediate radiative transfer variables of extremely high mixing ratio hydrometeor profile
at channel 150.0GHZ: $factor_{lower} = factor_{lower} = 10^{1}$.}
\label{fig:MWHSX2zi44}
\end{figure}

\begin{figure}[hbtp] 
\centering
\subfigure[Upward radiance, source term and extinction loss term]{
\centering
\includegraphics[width=0.70\textwidth]{./pdf/MWHSX/MWHSX3zi44up.pdf}
\label{subfig:MWHSX3zi44up}
}
\subfigure[Downward radiance, source term and extinction loss term]{
\centering
\includegraphics[width=0.70\textwidth]{./pdf/MWHSX/MWHSX3zi44do.pdf}
\label{subfig:MWHSX3zi44do}
}
\subfigure[Transmission of layers]{
\centering
\includegraphics[width=0.90\textwidth]{./pdf/MWHSX/MWHSX3tau.pdf}
\label{subfig:MWHSX3tau}
}
\caption{Intermediate radiative transfer variables of extremely high mixing ratio hydrometeor profile
at channel 183.31$\pm$7.0GHZ: $factor_{lower} = factor_{lower} = 10^{1}$.}
\label{fig:MWHSX3zi44}
\end{figure}

\clearpage
Insert page


\begin{figure}[hbtp] 
\centering
\subfigure[150.0GHZ]{
\centering
\includegraphics[width=0.62\textwidth]{./pdf/MWHSX/MWHSX2ABSP.pdf}
\label{subfig:MWHSX2ABSP}
}
\subfigure[183.31$\pm$7.0GHZ]{
\centering
\includegraphics[width=0.62\textwidth]{./pdf/MWHSX/MWHSX3ABSP.pdf}
\label{subfig:MWHSX3ABSP}
}
\caption{Total bulk scattering properties accounting for hydrometeors and gas at channels 150.0GHZ
and 183.31$\pm$7.0GHZ with extremely high mixing ratio hydrometeor profile: $factor_{lower} = factor_{lower} = 10^{1}$}
\label{fig:MWHSX23ABSP}
\end{figure}

\begin{figure}[hbtp] 
\centering
\includegraphics[width=0.68\textwidth]{./pdf/MWHSX/MWHSX2BT.pdf}
\caption{150.0GHZ brightness temperature aganist high and low adjusting factor between $10^{-2}$ and $10^{1}$ for four vertical 
inhomogeneity schemes of ice particle shapes.}
\label{fig:MWHSX2BT}
\end{figure}

\begin{figure}[hbtp] 
\centering
\includegraphics[width=0.8\textwidth]{./pdf/MWHSX/MWHSX2OVB.pdf}
\caption{Observed and simulated over-ocean brightness temperatures on 31 August 2018 in the region of Typhoon Feiyan at channel 150.0GHZ of MWHSX.
Simulations are generated by GRAPES RMFS forecast (10km operational run 2018083100z +4h). The satellite FY-3D overpassed the
tropical cyclone at 04:02 UTC}
\label{fig:MWHSX2OVB}
\end{figure}

\begin{figure}[hbtp] 
\centering
\includegraphics[width=0.68\textwidth]{./pdf/MWHSX/MWHSX3BT.pdf}
\caption{183.31$\pm$7.0GHZ brightness temperature aganist high and low adjusting factor between $10^{-2}$ and $10^{1}$ for four vertical 
inhomogeneity schemes of ice particle shapes.}
\label{fig:MWHSX3BT}
\end{figure}

\begin{figure}[hbtp] 
\centering
\includegraphics[width=0.8\textwidth]{./pdf/MWHSX/MWHSX3OVB.pdf}
\caption{Observed and simulated over-ocean brightness temperatures on 31 August 2018 in the region of Typhoon Feiyan at channel 183.31$\pm$7.0GHZ of MWHSX.
Simulations are generated by GRAPES RMFS forecast (10km operational run 2018083100z +4h). The satellite FY-3D overpassed the
tropical cyclone at 04:02 UTC}
\label{fig:MWHSX3OVB}
\end{figure}